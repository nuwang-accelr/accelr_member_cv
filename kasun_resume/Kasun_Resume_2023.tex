%%%%%%%%%%%%%%%%%%%%%%%%%%%%%%%%%%%%%%%%%
% Medium Length Professional CV
% LaTeX Template
% Version 3.0 (December 17, 2022)
%
% This template originates from:
% https://www.LaTeXTemplates.com
%
% Author:
% Vel (vel@latextemplates.com)
%
% Original author:
% Trey Hunner (http://www.treyhunner.com/)
%
% License:
% CC BY-NC-SA 4.0 (https://creativecommons.org/licenses/by-nc-sa/4.0/)
%
%%%%%%%%%%%%%%%%%%%%%%%%%%%%%%%%%%%%%%%%%

%----------------------------------------------------------------------------------------
%	PACKAGES AND OTHER DOCUMENT CONFIGURATIONS
%----------------------------------------------------------------------------------------

\documentclass[
	%a4paper, % Uncomment for A4 paper size (default is US letter)
	11pt, % Default font size, can use 10pt, 11pt or 12pt
]{resume} % Use the resume class

% \usepackage{ebgaramond} % Use the EB Garamond font
\usepackage{helvet}


%------------------------------------------------

\name{Kasun Buddhi} % Your name to appear at the top

\phone{+94773166850}
\linkedin{https://www.linkedin.com/in/kasun-dasanayaka/}
\email{sales@accelr.net}


% You can use the \address command up to 3 times for 3 different addresses or pieces of contact information
% Any new lines (\\) you use in the \address commands will be converted to symbols, so each address will appear as a single line.

\address{Email \\ sales@acceler.net} % Email

\address{WhatsApp \\ +94 (0)77 3166 850} % WhatsApp Number

\address{Linkedin \\ https://www.linkedin.com/in/kasun-dasanayaka/} % LinkedIn Profile

%----------------------------------------------------------------------------------------

\begin{document}


% \begin{tabularx}{\textwidth}{
% 	| >{\raggedright\arraybackslash}X 
% 	| >{\raggedleft\arraybackslash}X | }
% 	\hline
% 	{\huge\bf Kasun Buddhi} \\
% 	WhatsApp : Linkedin : Email
	
% 	& \raisebox{-\totalheight}{\includegraphics[width=0.3\textwidth]{logo.png}} \\
% 	\hline
% \end{tabularx} 

% \begin{tabularx}{\textwidth}{ |X|X| } 
% 	\hline
% 	cell3 & \multirow{3}{5cm}{Multiple row} \\ 
% 	cell6 &  \\ 
% 	cell9 &  \\ 
% 	\hline
% \end{tabularx}

% \begin{tabularx}{\textwidth}{
% 	 	 >{\raggedright\arraybackslash}X 
% 	 	 >{\raggedleft\arraybackslash}X  } 
% 	\smallskip
% 	{\huge\bf Kasun Buddhi} & 
% 	\multirow[c]{3}{*}{{\includegraphics[width=0.25\textwidth]{logo.png}}}\\ 
% 	WhatsApp : Linkedin : Email & \\
% \end{tabularx}

%----------------------------------------------------------------------------------------
%	TECHNICAL STRENGTHS SECTION
%----------------------------------------------------------------------------------------

\begin{rSection}{Technical Strengths}

	\def\arraystretch{1.5}

	\begin{tabular}{ l l}
		\textbf{Expertise} & \emph{Design Verification, RTL/Digital Design, Embedded Systems} \\
		\textbf{Programming Languages} & \emph{C, Python, SystemVerilog} \\
		\textbf{Tools and Frameworks} & \emph{Siemens Questa, UVM, Xilinx Vivado} \\ 
		\textbf{Languages} & \emph{Sinhala-Native, English-Excellent} \\
	\end{tabular}

\end{rSection}

%----------------------------------------------------------------------------------------
%	EDUCATION SECTION
%----------------------------------------------------------------------------------------

\begin{rSection}{Education}

	\textbf{Skill Surf, University of Moratuwa, Sri Lanka} \hfill \textit{2023} \\ 
	Certificate Course - SystemVerilog for ASIC/FPGA Design \& Simulation \\
	% Minor in Linguistics \smallskip \\
	% Member of Eta Kappa Nu \\
	% Member of Upsilon Pi Epsilon \\
	Status : Completed

	\textbf{University of Moratuwa, Sri Lanka} \hfill \textit{2017 - 2021} \\ 
	B.S. (Hons) Electrical Engineering \\
	% Minor in Linguistics \smallskip \\
	% Member of Eta Kappa Nu \\
	% Member of Upsilon Pi Epsilon \\
	Overall GPA: 3.01/4.2 \\
	Status: Second Lower
	
\end{rSection}

%----------------------------------------------------------------------------------------
%	WORK EXPERIENCE SECTION
%----------------------------------------------------------------------------------------

\begin{rSection}{Experience}

	\begin{rSubsectionX}{ACCELR}{www.accelr.lk}{Electronic Engineer}{Jan 2023 - Present}
		\item Study of the PULP (RISC-V) platform and its verification environment. \href{https://github.com/pulp-platform/pulpissimo}{Pulpisimo on GitHub Repo}
		\item In the process of building VIPs and a reusable UVM verification environment for PULP uDMA peripheral IPs starting with the UART uDMA IP.
		\item Design and implementation of a packet-oriented AXI4-stream verification environment using the Xilinx beat-oriented AXI4-stream VIP.
	\end{rSubsectionX}

	\begin{rSubsectionSimpleX}{Lanka Electric Company, SL}{leco.lk}{Intern - Electrical Engineer}{May 2021 - June 2021}
	\end{rSubsectionSimpleX}

	\begin{rSubsectionSimpleX}{Ceylon Electricity Board, SL}{www.ceb.lk}{Intern - Electrical Engineer}{Jan 2021 - April 2021}
	\end{rSubsectionSimpleX}

\end{rSection}

%----------------------------------------------------------------------------------------
%	PROJECTS
%----------------------------------------------------------------------------------------

\begin{rSection}{Projects}

	\textbf{PULP uDMA UVM verification environment} \\
	At ACCELR we have embarked on the development of a truly open-source UVM verification environment that can be used to verify peripheral IPs that come with the PULP RISC-V. We are currently in the process of building such an environment for the uDMA UART IP core. The agents and framework that we are building can easily be adapted to verify other uDMA IP cores - i.e. SPI, I2C, HyperRAM, HyperFlash, etc. Our chief contribution will be a UVM agent that mocks the PULP uDMA code, thereby allowing anyone to easily verify uDMA IPs and add them to the PULP system with ease in the future.

	\textbf{Packet-based AXI4-Stream agent} \\
	Implementation of a packet-based AXI4-Stream agent (non-UVM) and verification environment by leveraging the free Xilinx AXI4-stream VIP.

	\textbf{Domestic floor cleaning robot} \\
	As the final year undergraduate project, an intelligent domestic floor cleaning robot was designed and simulations were carried out with Webot simulator. The robot incorporated vacuum cleaning using multiple sensors and manipulators. The design was then modeled using Blender (a 3D computer graphics software tool).

	\textbf{Electric door controller} \\
	As an undergraduate project, developed an electrical sliding door. An IoT device (Arduino) was used as the controller as well as an Android app (Kotlin) used for remote control of the door.

\end{rSection}

%----------------------------------------------------------------------------------------
%	RESEARCH
%----------------------------------------------------------------------------------------

%\begin{rSection}{Research}

	%Section content\ldots

%\end{rSection}

%----------------------------------------------------------------------------------------
%	PUBLICATIONS
%----------------------------------------------------------------------------------------

%\begin{rSection}{Publications}

	%Section content\ldots

%\end{rSection}

%----------------------------------------------------------------------------------------
%	PROFESSIONAL AFFILIATIONS
%----------------------------------------------------------------------------------------

\begin{rSection}{Professional Affiliations}

	\textbf{Institution of Engineers, Sri Lanka (IESL)} \hfill \textit{Since 2019} \\ 
	Status : Student Member \\
	Membership No.: S-26740

	\textbf{IEEE} \hfill \textit{Since 2019} \\ 
	Status : Student Member \\
	% Membership No.: xxxxxxxx

\end{rSection}

%----------------------------------------------------------------------------------------
%	ACHIEVEMENTS
%----------------------------------------------------------------------------------------

% \begin{rSection}{Achievement}

% 	\textbf{1\textsuperscript{st} place - Black Belt Male-Kumite event} \hfill \textit{2016} \\ 
% 	Gained 1\textsuperscript{st} place at the Black Belt Male-Kumite event organized by Kensho Karate International Sri Lanka Karate Do Federation (approved by the Ministry of Sports).

% 	\textbf{Best mini project for power electronics} \hfill \textit{2017} \\ 
% 	Best mini project for power electronics at the Department of Electrical and Electronics Engineering, South Eastern University, Sri Lanka.

% 	\textbf{Best student award - SMIDF} \hfill \textit{2009} \\ 
% 	Best student award for \emph{Computer Hardware} awarded by the SMIDF (Small \& Medium Industrial Development Foundation), Kururnegala.

% \end{rSection}

%----------------------------------------------------------------------------------------
%	LEADERSHIP AND TEAMWORK
%----------------------------------------------------------------------------------------

%\begin{rSection}{RESEARCH}

	%Section content\ldots

%\end{rSection}

%----------------------------------------------------------------------------------------
%	EXTRA CURRICULAR ACTIVITIES
%----------------------------------------------------------------------------------------

% \begin{rSection}{Extra Curricular Activities}

% 	\textbf{Member of University Badminton team} \hfill \textit{2014 - 2018}

% 	\textbf{Member of University Taekwondo team} \hfill \textit{2014 - 2018}

% 	\textbf{Member of University Media club} \hfill \textit{2014 - 2018}

% \end{rSection}








%----------------------------------------------------------------------------------------
%	TECHNICAL STRENGTHS SECTION
%----------------------------------------------------------------------------------------

% \begin{rSection}{Technical Strengths}

% 	\begin{tabular}{@{} >{\bfseries}l @{\hspace{6ex}} l @{}}
% 		Computer Languages & Prolog, Haskell, AWK, Erlang, Scheme, ML \\
% 		Protocols \& APIs & XML, JSON, SOAP, REST \\
% 		Databases & MySQL, PostgreSQL, Microsoft SQL \\
% 		Tools & SVN, Vim, Emacs
% 	\end{tabular}

% \end{rSection}

%----------------------------------------------------------------------------------------
%	EXAMPLE SECTION
%----------------------------------------------------------------------------------------

%\begin{rSection}{Section Name}

	%Section content\ldots

%\end{rSection}

%----------------------------------------------------------------------------------------

\end{document}
