%%%%%%%%%%%%%%%%%%%%%%%%%%%%%%%%%%%%%%%%%
% Medium Length Professional CV
% LaTeX Template
% Version 3.0 (December 17, 2022)
%
% This template originates from:
% https://www.LaTeXTemplates.com
%
% Author:
% Vel (vel@latextemplates.com)
%
% Original author:
% Trey Hunner (http://www.treyhunner.com/)
%
% License:
% CC BY-NC-SA 4.0 (https://creativecommons.org/licenses/by-nc-sa/4.0/)
%
%%%%%%%%%%%%%%%%%%%%%%%%%%%%%%%%%%%%%%%%%

%----------------------------------------------------------------------------------------
%	PACKAGES AND OTHER DOCUMENT CONFIGURATIONS
%----------------------------------------------------------------------------------------

\documentclass[
	%a4paper, % Uncomment for A4 paper size (default is US letter)
	11pt, % Default font size, can use 10pt, 11pt or 12pt
]{./assets/resume} % Use the resume class

% \usepackage{ebgaramond} % Use the EB Garamond font
\usepackage{helvet}
\usepackage{color, soul}


%------------------------------------------------

\name{Malsha Fernando} % Your name to appear at the top

\phone{+94773166850}
\linkedin{https://www.linkedin.com/in/malsha-pinibidu-fernando}
\email{sales@accelr.net}


% You can use the \address command up to 3 times for 3 different addresses or pieces of contact information
% Any new lines (\\) you use in the \address commands will be converted to symbols, so each address will appear as a single line.

\address{Email \\ sales@acceler.net} % Email

\address{WhatsApp \\ +94 (0)77 3166 850} % WhatsApp Number

\address{Linkedin \\ https://www.linkedin.com/in/malsha-pinibidu-fernando} % LinkedIn Profile

%----------------------------------------------------------------------------------------

\begin{document}


% \begin{tabularx}{\textwidth}{
% 	| >{\raggedright\arraybackslash}X 
% 	| >{\raggedleft\arraybackslash}X | }
% 	\hline
% 	{\huge\bf Kasun Buddhi} \\
% 	WhatsApp : Linkedin : Email
	
% 	& \raisebox{-\totalheight}{\includegraphics[width=0.3\textwidth]{logo.png}} \\
% 	\hline
% \end{tabularx} 

% \begin{tabularx}{\textwidth}{ |X|X| } 
% 	\hline
% 	cell3 & \multirow{3}{5cm}{Multiple row} \\ 
% 	cell6 &  \\ 
% 	cell9 &  \\ 
% 	\hline
% \end{tabularx}

% \begin{tabularx}{\textwidth}{
% 	 	 >{\raggedright\arraybackslash}X 
% 	 	 >{\raggedleft\arraybackslash}X  } 
% 	\smallskip
% 	{\huge\bf Kasun Buddhi} & 
% 	\multirow[c]{3}{*}{{\includegraphics[width=0.25\textwidth]{logo.png}}}\\ 
% 	WhatsApp : Linkedin : Email & \\
% \end{tabularx}

%----------------------------------------------------------------------------------------
%	TECHNICAL STRENGTHS SECTION
%----------------------------------------------------------------------------------------

\begin{rSection}{Technical Strengths}

	\def\arraystretch{1.5}

	\begin{tabular}{ l l}
		\textbf{Expertise} & \emph{RTL/Digital Design, ML \& Deep Learning, Embedded Systems} \\
		\textbf{Programming Languages} & \emph{C++, C, Verilog, Python, R, SQL} \\
		\textbf{Tools and Frameworks} & \emph{Xilinx Vivado, Matlab \& Simulink, LabVIEW, MySQL} \\ 
		\textbf{Languages} & \emph{Sinhala-Native, English-Excellent} \\
	\end{tabular}

\end{rSection}

%----------------------------------------------------------------------------------------
%	EDUCATION SECTION
%----------------------------------------------------------------------------------------

\begin{rSection}{Education}

	\textbf{University of Sri Jayewardenepura, Sri Lanka} \hfill \textit{2018 - 2022} \\ 
	B.S. (Hons) Electrical \& Electronic Engineering \\
	Minor in Telecommunication Engineering \smallskip \\
	% Member of Eta Kappa Nu \\
	% Member of Upsilon Pi Epsilon \\
	Overall GPA: 3.22/4.0 \\
	Status: Second Class Honours
	
\end{rSection}

%----------------------------------------------------------------------------------------
%	WORK EXPERIENCE SECTION
%----------------------------------------------------------------------------------------

\begin{rSection}{Experience}

	\begin{rSubsectionX}{ACCELR}{www.accelr.lk}{software Engineer}{Nov 2022 - Present}
		\item Conducting a study on the IPC efficiency in using the RISC-V compression instruction for network packet forwarding applications. Experiments are being carried out using the marss-riscv : micro-architectural system simulator for RISC-V. Our intent here is to validate the claims made in the  research paper titled "A Compression Instruction Set Design based on RISC-V for Network Packet Forwarding" by researchers from the National University of Defense Technology, Changsha, China. \href{https://iopscience.iop.org/article/10.1088/1742-6596/1026/1/012001/pdf}{paper} 
		\item Extensive study of various compression algorithms, including huffman coding, for demanding applications in text, image, audio, and video compression.
	\end{rSubsectionX}

	\begin{rSubsectionX}{Analog Inference}{www.analog-inference.com}{Engineering Consultant}{Nov 2022 - Present}
		\item member of the Analog Inference back-end software development team primarily responsible for mapping and optimizing neural networks on the AI data flow accelerator hardware.
        \item Responsible for optimizing CNNs such as Yolo, Resnet, and FCN for Analog Inference hardware. 
        \item Gained extensive experience on the low level and high level hardware simulators and runtime environment.
        \item Carried out mappings of quantized neural network models to fit Analog Inference data flow hardware architecture while addressing memory constraints and optimizing for maximum throughput.
        \item Developed software tools for post-silicon validation (PSV) of the Analog Inference chip, including testing and evaluation
        \item Involved in development of "spreadsheet-automated mapper" software tool to optimize models and detect anomalies and violations in the neural network mappings.
	\end{rSubsectionX}

	\pagebreak

	\begin{rSubsectionX}{Huawei Technologies, SL}{www.huawei.com}{Intern - Electronic Engineer}{Nov 2021 - May 2022}
		\item Conducted a PoC for the Road Developed Authority (RDA) - central expressway CCTV project. This entailed the testing personal and vehicle image processing algorithms such as facial recognition and loitering detection.
		\item Engaged with customers on WLAN planning projects, providing technical support and solutions. 
		\item Presented emerging trends in Internet of Things (IoT) at the Mini MWC global event, showcasing Huawei's innovations in the field.
	\end{rSubsectionX}

\end{rSection}

%----------------------------------------------------------------------------------------
%	PROJECTS
%----------------------------------------------------------------------------------------

\begin{rSection}{Projects}

	\textbf{Efficient hardware architecture for Reinforcement Learning} \\
	Final Year Project of B.Sc. degree program. The primary goal of this project is to design and implement a novel efficient hardware architecture for reinforcement learning - i.e. the Q learning algorithm - by optimizing the tradeoff between power consumption, throughput, and silicon area. Designed using Verilog HDL and implemented on Artix 7 FPGA. Results were obtained using Vivado 2019.1 EDA tool with verification support of MATLAB simulink via a Vivado system generator for DSP.

	\textbf{Rehabilitation support wheelchair} \\
	Designed a prototype robotic wheelchair with an adaptive robotic arm for patients undergoing rehabilitation, incorporating features to measure vital parameters such as pulse rate, oxygen saturation, and blood pressure.

\end{rSection}

%----------------------------------------------------------------------------------------
%	RESEARCH
%----------------------------------------------------------------------------------------

%\begin{rSection}{Research}

	%Section content\ldots

%\end{rSection}

%----------------------------------------------------------------------------------------
%	PUBLICATIONS
%----------------------------------------------------------------------------------------

%\begin{rSection}{Publications}

	%Section content\ldots

%\end{rSection}

%----------------------------------------------------------------------------------------
%	PROFESSIONAL AFFILIATIONS
%----------------------------------------------------------------------------------------

\begin{rSection}{Professional Affiliations}

	\textbf{Institution of Engineers, Sri Lanka (IESL)} \hfill \hl{\textit{Since 2023}} \\ 
	Status : Associate Member %\\
	% Membership No.: AM-30835

	% \textbf{Engineering Council Sri Lanka (ECSL)} \hfill \hl{\textit{Since 20??}} \\ 
	% Status : Associate Engineer (AEng) %\\
	% \hl{Shall we remove this??}
	% % Membership No.: AM-30835

\end{rSection}

%----------------------------------------------------------------------------------------
%	ACHIEVEMENTS
%----------------------------------------------------------------------------------------

\begin{rSection}{Certificates \& Achievements}

	\textbf{Winner, Huawei ICT Competition - Network \& Cloud track} \hfill \textit{2022} \\ 
	Went on to represented Sri Lanka at the Huawei ICT Competition 2020 APAC Regional Finals \\
	
	\textbf{4\textsuperscript{th} place - Robotics and Embedded Systems Category - Innovate Sri Lanka Exhibition \& Competition} \hfill \textit{2019} \\ 
	Developed a moveable CNC writer and won 4th place at the competition \\

	\textbf{HCIA Video Conference Engineering - Huawei Certification} \hfill \textit{2009} \\ 
	Valid through 2025

\end{rSection}

%----------------------------------------------------------------------------------------
%	LEADERSHIP AND TEAMWORK
%----------------------------------------------------------------------------------------

\begin{rSection}{Leadership \& Teamwork}

	\textbf{Editor - E\textsuperscript{2} Club - USJ} \hfill \textit{2019 - 2021} \\ 
	As the editor of  E\textsuperscript{2} Club, organized seminars, webinars, tech talks, technical workshops for USJ students.
	
	\textbf{Analysis of images from pan-STARRS} \hfill \textit{2019 - 2021} \\
	Contributions to observations of near-Earth objects and main belt asteroids by participating in the analysis of images from pan-STARRS (The campaign was organized by IASC and NASA).

	\textbf{SEDS Committee Member - USJ chapter} \hfill \textit{2019 - 2022} \\ 
	Member of Students for the Exploration \& Development of Space (SEDS), Sri Lanka.

	\textbf{“Sith Paura” Program - USJ} \hfill \textit{2019} \\
	Volunteered for the “Sith Paura” program organized by Faculty of Engineering, USJ. The program supports students in underprivileged areas by conducting maths seminars for {O/L} and {A/L} as well as fundraisers for worthy causes such as building libraries.

\end{rSection}

%----------------------------------------------------------------------------------------
%	EXTRA CURRICULAR ACTIVITIES
%----------------------------------------------------------------------------------------

% \begin{rSection}{Extra Curricular Activities}

% 	\textbf{Member of University Badminton team} \hfill \textit{2014 - 2018}

% 	\textbf{Member of University Taekwondo team} \hfill \textit{2014 - 2018}

% 	\textbf{Member of University Media club} \hfill \textit{2014 - 2018}

% \end{rSection}

%----------------------------------------------------------------------------------------
%	TECHNICAL STRENGTHS SECTION
%----------------------------------------------------------------------------------------

% \begin{rSection}{Technical Strengths}

% 	\begin{tabular}{@{} >{\bfseries}l @{\hspace{6ex}} l @{}}
% 		Computer Languages & Prolog, Haskell, AWK, Erlang, Scheme, ML \\
% 		Protocols \& APIs & XML, JSON, SOAP, REST \\
% 		Databases & MySQL, PostgreSQL, Microsoft SQL \\
% 		Tools & SVN, Vim, Emacs
% 	\end{tabular}

% \end{rSection}

%----------------------------------------------------------------------------------------
%	EXAMPLE SECTION
%----------------------------------------------------------------------------------------

%\begin{rSection}{Section Name}

	%Section content\ldots

%\end{rSection}

%----------------------------------------------------------------------------------------

\end{document}
