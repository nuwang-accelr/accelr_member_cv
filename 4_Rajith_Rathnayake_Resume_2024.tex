%%%%%%%%%%%%%%%%%%%%%%%%%%%%%%%%%%%%%%%%%
% Medium Length Professional CV
% LaTeX Template
% Version 3.0 (December 17, 2022)
%
% This template originates from:
% https://www.LaTeXTemplates.com
%
% Author:
% Vel (vel@latextemplates.com)
%
% Original author:
% Trey Hunner (http://www.treyhunner.com/)
%
% License:
% CC BY-NC-SA 4.0 (https://creativecommons.org/licenses/by-nc-sa/4.0/)
%
%%%%%%%%%%%%%%%%%%%%%%%%%%%%%%%%%%%%%%%%%

%----------------------------------------------------------------------------------------
%	PACKAGES AND OTHER DOCUMENT CONFIGURATIONS
%----------------------------------------------------------------------------------------

\documentclass[
%a4paper, % Uncomment for A4 paper size (default is US letter)
11pt, % Default font size, can use 10pt, 11pt or 12pt
]{./assets/resume} % Use the resume class
% \usepackage{ebgaramond} % Use the EB Garamond font
\usepackage{helvet}

%----------------------------------------------------------------------------------------
%	NAME SECTION
%----------------------------------------------------------------------------------------

\name{Rajith Rathnayake} % Your name to appear at the top

\phone{+94773166850}
\linkedin{https://www.linkedin.com/in/rajith-rathnayake-b11598111/}
\email{sales@accelr.net}

% You can use the \address command up to 3 times for 3 different addresses or pieces of contact information
% Any new lines (\\) you use in the \address commands will be converted to symbols, so each address will appear as a single line.

% \address{Email \\ sales@acceler.net} % Email

% \address{WhatsApp \\ +94 (0)71 6487 689} % WhatsApp Number

% \address{Linkedin \\ https://www.linkedin.com/in/kavinga-upul-ekanayaka/} % LinkedIn Profile

%------------------------------------------------

\begin{document}

%----------------------------------------------------------------------------------------
%	TECHNICAL STRENGTHS SECTION
%----------------------------------------------------------------------------------------
\begin{rSection}{Technical Strengths}
	
	\def\arraystretch{1.5}
	
	\begin{tabular}{p{2.0in} p{4.5in}}
		\textbf{Expertise} & \emph{RTL design, Static Verification, Simulation, Emulation, Prototyping} \\
		\textbf{Programming Languages} & \emph{Verilog, SystemVerilog, VHDL, Python, C,  Java, C\#} \\
		\textbf{Tools and Frameworks} & \emph{VCS, VC Spyglass, VC Formal, Xilinx Vivado, Matlab} \\ 
		\textbf{Languages} & \emph{Sinhala-Native, English-Excellent} \\
	\end{tabular}
	
\end{rSection}

%----------------------------------------------------------------------------------------
%	EDUCATION SECTION
%----------------------------------------------------------------------------------------

\begin{rSection}{Education}
	
	\textbf{University of Westminster, UK} \hfill \textit{2019 - 2020} \\ 
	M.Sc (Hons) in Advanced Software Engineering \\
	Status : Distinction Pass

	\textbf{University of Moratuwa, Sri Lanka} \hfill \textit{2012 - 2017} \\ 
	B.Sc (Hons) in Electronic \& Telecommunication Engineering \\
	Status : Second Class Upper (GPA : 3.51 / 4.20)
	
\end{rSection}

%----------------------------------------------------------------------------------------
%	EXPERIENCE SECTION
%----------------------------------------------------------------------------------------

\begin{rSection}{Experience}
	%X-EPIC
	\begin{rSubsectionX}{X-EPIC}{www.x-epic.com}{Senior Product Engineer Level 1}{May 2022 - Jul 2024}
		\item Product validation of simulation, emulation and prototyping tools 
		\item RTL simulator validation: Validated the RTL simulator Galaxsim according to the SystemVerilog LRM covering assertions, coverage, and other SystemVerilog Constructs. 
		\item Emulator/Prototyping Validation: Validated emulation and prototyping tools HuaEmu and HuaPro with multiple large designs and covering memory related testing, net-list optimizations, performance testing, etc. 
		\item RISC-V SoC validation: Conducted extensive validation of the in-house tools using \textit{OpenPiton}, \textit{Open C910} and \textit{HummingBird} RISCV SoCs on both simulation and emulation platforms, ensuring accurate functionality and performance 
		\item Design bank implementation: Implemented a design bank for in-house quality of result and performance testing using open-source Verilog/SystemVerilog designs, facilitating improved testing and validation processes. 
		\item Knowledge sharing sessions: Conducted knowledge-sharing sessions for peers related to emulation and prototyping technologies, enhancing team expertise and collaborative learning.
	\end{rSubsectionX}
	%Synopsys
	\begin{rSubsectionX}{Synopsys}{www.synopsys.com}{Application Engineer}{Oct 2018 - May 2022}
		\item Verification of tool functionality of EDA tools.
		\item RTL Simulator Validation: Designing testbenches in hardware descriptive languages (Verilog/VHDL) to digital electronic design RTL and use SystemVerilog Assertions to verify assertion and assumption properties using simulator tools such as VCS. 
		\item CDC Validation: Verifying the clock domain crossing rule functionality of complex customer digital electronic designs with the Synopsys static verification tools to identify the metastability, glitch, convergence and functionality failures. 
		\item RDC Validation: Verifying the reset domain crossing rule functionality of complex customer digital electronic designs with the Synopsys static verification tools to identify the metastability, glitch, convergence and functionality failures. 
		\item Ownership: Product release owner and documentation owner of the VC RDC team. Responsible for managing the release patches with good quality and quantity. Designing automation scripts using Perl and Python to enhance the quality of patch validation.
	\end{rSubsectionX}
	%Dialog
	\begin{rSubsectionX}{Dialog Axiata PLC, SL}{www.dialog.lk}{DevOps Engineer}{Apr 2017 - Oct 2018}
		\item Bridge the gap between development and IT operations in Dialog Axiata Software Products.
		\item Software project management using Atlassian JIRA Software.
		\item IT service management (ITSM) Using Atlassian JIRA Service Desk.
		\item Maintaining ATG Oracle Web Commerce platform for Axiata Web commerce products.
	\end{rSubsectionX}
	%wavecomputing
	\begin{rSubsectionX}{Wave Computing, US}{www.wavecomp.ai}{Intern}{Feb 2016 - Apr 2016}
		\item RTL functional verification
		\item Timing analysis and verification using nano time TCL scripting
	\end{rSubsectionX}
	%paraqum
	\begin{rSubsectionX}{ParaQum Technologies, SL}{www.wavecomp.ai}{Intern}{Oct 2015 - Jan 2016}
		\item Digital system designing and simulations
		\item Functional and timing verification of ASIC designs
	\end{rSubsectionX}

\end{rSection}

%----------------------------------------------------------------------------------------
%	RESEARCH SECTION
%----------------------------------------------------------------------------------------
% \begin{rSection}{Research}
% 	% CEP 
% 	\textbf{Hardware Implementation of a Complex Event Processor} \hfill \textit{Apr 2012 - Apr 2013}\\
% 	A hardware accelerated complex event processor (CEP) platform was designed and implemented on FPGA with reference to WSO2 siddhi software CEP platform.
% 	The design is highly parameterized to enhance the flexibility, scalability and compatibility with the software platform.
% 	Achieved more than 10x performance than its software counterpart verified using a real world dataset.

	
% 	% cloud computing
% 	\textbf{Hardware Acceleration for Cloud computing architectures} \hfill \textit{Apr 2013 - Oct 2015}\\
% 	Thorough analysis of cloud computing architecture helped to find out Network virtualization as the major bottleneck.
% 	Parallel processing techniques were used to improve the QoS of network virtualization using a hardware switch fabric designed in FPGA

	
% \end{rSection}

%----------------------------------------------------------------------------------------
%	PUBLICATION SECTION
%----------------------------------------------------------------------------------------
% \begin{rSection}{Publications}

% 	% international conference
% 	\textbf{IEEE Conference Paper } \hfill  \textit{October 2014}\\
% 	%\begin{quote}
% 	Ekanayaka, K.U.B.; Pasqual, A., ``FPGA based custom accelerator architecture framework for complex event processing," \emph{TENCON 2014 - 2014 IEEE Region 10 Conference} , vol., no., pp.1,6, 22-25 Oct. 2014
% 	%\end{quote}

% \end{rSection}
 
%----------------------------------------------------------------------------------------
%	ACHIEVEMENTS SECTION
%----------------------------------------------------------------------------------------
\begin{rSection}{Achievements}

	\textbf{International Robotics Challenge (IRC), IIT Bombay} \hfill \textit{2015}\\
	4th place, University Category

	\textbf{Sri Lanka Robotics Challenge, University of Moratuwa} \hfill \textit{2014}\\
	Champions, University Category

	\textbf{Sri Lanka Robotics Challenge, University of Moratuwa} \hfill \textit{2013}\\
	4th place, Open Category

	\textbf{Axiata Digital Jam 2017} \hfill \textit{2017}\\
	4th place from online activation and engagement event with over 9,000 employees in six countries.

\end{rSection}
%----------------------------------------------------------------------------------------
%	HOUNORS AND AWARDS SECTION
%---------------------------------------------------------------------------------------- 
% \begin{rSection}{Honors and Awards}

% 	%sri lankan physics olympiad
% 	\textbf{Bronze Medal at Sri Lankan Physics Olympiad} \hfill \textit{Apr 2006}\\
% 	Achieved a bronze medal at SLPHO organized by Institute of Physics, University of Colombo, Sri Lanka.


% \end{rSection}

%----------------------------------------------------------------------------------------
%	PROJECTS SECTION
%----------------------------------------------------------------------------------------
\begin{rSection}{Projects}

	\textbf{RISC-V IoT specialized processor} \hfill \textit{2017}\\
	Developed an ASIC design for an IoT processor based on the RISC-V architecture. Mostly concern about the power and area requirements.

	\textbf{Processor design using FPGA} \hfill \textit{2017}\\
	Implemented a simple Intel 8085 like processor using FPGA.

	\textbf{Image encryption using AES Algorithm} \hfill \textit{2017}\\
	Implemented a core that is capable of encrypting and decrypting a given image using AES 128-bit algorithm. 

	\textbf{Robot to complete a given task (robot competition)} \hfill \textit{2017}\\
	Functionalities include line following, Node and object detecting, grid solving (calculating the shortest path), gripping box

	\textbf{Remote Controlled Robot} \hfill \textit{2017}\\
	We built a remote-controlled robot that can be controlled wirelessly and is capable of going over uneven surfaces and shooting an object.

	\textbf{Color Separator} \hfill \textit{2017}\\
	This robot is capable of correctly identifying red, green and blue color papers from a bundle of papers, arrange them according to their colors and take them to the correct places.

\end{rSection}

%----------------------------------------------------------------------------------------
%	PROFFESSIONAL AFFILIATIONS SECTION
%----------------------------------------------------------------------------------------
\begin{rSection}{Professional Affiliations}
	% IEEE
	\textbf{Institute of Electrical and Electronics Engineers (IEEE)} \hfill \textit{Since 2017}\\
	Status : Member
	
\end{rSection}

%----------------------------------------------------------------------------------------
%	LEADERSHIP AND TEAMWORK SECTION
%----------------------------------------------------------------------------------------

% \begin{rSection}{Leadership and Teamwork}

% % DVCON
% \textbf{Sri Lanka Liason Chair} \hfill \textit{2022 - 2023} \\
% Design and Verification Conference (DVCon-India)


% % IEEE vTools
% \textbf{Member} \hfill \textit{2018} \\
% IEEE MGA vTools Committee


% % IEEE Region 10
% \textbf{Chairman} \hfill \textit{2015} \\
% IEEE Region 10 (Asia/Pacific) Congress - Colombo, Sri Lanka


% % IEEE R10 PAC
% \textbf{Member} \hfill \textit{2023} \\
% IEEE Region 10 (Asia/Pacific) Professional Activities Committee

% % IEEE sri lanka section
% \textbf{Assistant Treasurer} \hfill \textit{2015 - 2019}\\
% IEEE Sri Lanka section

% % Toastmasters
% \textbf{President} \hfill \textit{2015 - 2016} \\
% University of Moratuwa Toastmasters club

% % ICIAfS 
% \textbf{Publicity Chair} \hfill \textit{2014} \\
% 7\textsuperscript{th} IEEE International Conference on Information and Automation for Sustainability (ICIAfS)

% \end{rSection}

\end{document}