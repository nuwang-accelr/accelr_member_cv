%%%%%%%%%%%%%%%%%%%%%%%%%%%%%%%%%%%%%%%%%
% Medium Length Professional CV
% LaTeX Template
% Version 3.0 (December 17, 2022)
%
% This template originates from:
% https://www.LaTeXTemplates.com
%
% Author:
% Vel (vel@latextemplates.com)
%
% Original author:
% Trey Hunner (http://www.treyhunner.com/)
%
% License:
% CC BY-NC-SA 4.0 (https://creativecommons.org/licenses/by-nc-sa/4.0/)
%
%%%%%%%%%%%%%%%%%%%%%%%%%%%%%%%%%%%%%%%%%

%----------------------------------------------------------------------------------------
%	PACKAGES AND OTHER DOCUMENT CONFIGURATIONS
%----------------------------------------------------------------------------------------

\documentclass[
	%a4paper, % Uncomment for A4 paper size (default is US letter)
	11pt, % Default font size, can use 10pt, 11pt or 12pt
]{./../assets/resume} % Use the resume class

% \usepackage{ebgaramond} % Use the EB Garamond font
\usepackage{helvet}


%------------------------------------------------

\name{Binu Amaratunga} % Your name to appear at the top

\phone{+94773166850}
\linkedin{https://www.linkedin.com/in/bamaratunga/}
\email{sales@accelr.net}

% You can use the \address command up to 3 times for 3 different addresses or pieces of contact information
% Any new lines (\\) you use in the \address commands will be converted to symbols, so each address will appear as a single line.
\address{Email \\ sales@acceler.net} % Email
\address{WhatsApp \\ +94 (0)77 3166 850} % WhatsApp Number
\address{Linkedin \\ https://www.linkedin.com/in/bamaratunga/} % LinkedIn Profile

%----------------------------------------------------------------------------------------

\begin{document}

%----------------------------------------------------------------------------------------
%	TECHNICAL STRENGTHS SECTION
%----------------------------------------------------------------------------------------

\begin{rSection}{Technical Strengths}
	\def\arraystretch{1.3}
	\begin{tabular}{ll}
		\textbf{Expertise} & \emph{Software Development, High Performance Computing, }\\
                           & \emph{Hardware Acceleration, Compiler Development} \\
		\textbf{Programming Languages} & \emph{C++, C, Python, Verilog, SystemVerilog, Shell scripting} \\
		\textbf{Other} & \emph{MPI, OpenMP, CUDA, GDB, GNU Make, CMake, Git, Matlab} \\ 
		\textbf{Languages} & \emph{Sinhala-Native, English-Fluent, German-Elementary} \\
	\end{tabular}

\end{rSection}

%----------------------------------------------------------------------------------------
%	EDUCATION SECTION
%----------------------------------------------------------------------------------------

\begin{rSection}{Education}

	\textbf{Technical University of Munich, Germany} \hfill \textit{Oct 2019 - Present} \\ 
	\textit{\textbf{M. Sc. (Hons)} Computational Science and Engineering}  \\
    {\small Elite Graduate Program of Bavarian Graduate School of Computational Engineering (BGCE) \\
    Expected: December 2023 \\
    Current GPA: 1.8/1.0 (German grading system)}

	\textbf{University of Moratuwa, Sri Lanka} \hfill \textit{Jul 2010 - Apr 2015} \\ 
	\textit{\textbf{B. Sc. (Hons)} Electronic and Telecommunication Engineering} \\
	{\small Second Class - Upper Division \\
    Overall GPA: 3.55/4.20}

    \textbf{Chartered Institute of Management Accountants (CIMA), UK} \hfill \textit{Mar 2010 - Feb 2019} \\ 
    \textit{\textbf{CIMA Adv. Dip.} Management Accounting}
    % {\small Exam completed}

    \textbf{Nalanda College, Sri Lanka} \hfill
    \textit{August 2009}\\
    \textit{\textbf{G.C.E. A/L} Physical Science stream} \\
    {\small National Rank: \textbf{3rd} (Out of $\sim$200,000 candidates)} \\
    {\small Z-Score: 3.202}
	
\end{rSection}

%----------------------------------------------------------------------------------------
%	WORK EXPERIENCE SECTION
%----------------------------------------------------------------------------------------

\begin{rSection}{Experience}

	\begin{rExperienceOnePos} {Synopsys Inc., Sri  Lanka} {www.synopsys.com}
    {R\&D Engineer, Senior I} {Sep 2021 - Aug 2023}
        \item
        Worked on the Compiler Front-end tool which performs equivalence checking between reference RTL and synthesized netlist using Simulation and Formality. The tool was written in C++ and TCL.
        \item
        Worked with the Intermediate Representation (IR) for Verilog and SystemVerilog for developments in the decompiled reference and test-bench generation code which adds constructs required for the simulation flow.
        \item
        Worked on migration of names databases to the latest (database for mapping between reference RTL names and equivalent synthesized netlist names) for Formality flow.
        \item
        Worked on feature additions, enhancements, debugging, bug fixing and performance improvements of the tool.
	\end{rExperienceOnePos}

	\begin{rExperienceOnePos} {Infineon Technologies AG, Germany} {www.infineon.com}
    {Project Intern} {May 2020 - Apr 2021}
        \item
        Developed a Netlist Compiler for an Analog Circuit Simulator from scratch.
        \item
        The compiler was written in modern C++. Parsing Expression Grammar Template Library (PEGTL) was used in the parser.
        \item
        Included parsing, abstract syntax tree generation, expansion of hierarchical structures, name resolution, semantic analysis and certain optimizations.
        \item
        Designed and developed a language agnostic Intermediate Representation (IR) to represent netlists written in mixed language grammars (both case-sensitive and case-insensitive).
        \item
        Came up with and developed an efficient algorithm to achieve name resolution in constant time-complexity.
	\end{rExperienceOnePos}

\newpage

	\begin{rExperienceOnePos} {Technical University of Munich, Germany} {www.tum.de}
    {Teaching Assistant}{Oct 2020 - Mar 2021}
		\item
        Teaching assistant for the exercise sessions of Advanced Programming in C++ Masters course in Winter Semester 2020/21.
	\end{rExperienceOnePos}

	\begin{rExperienceThreePos}{Wave computing, USA} {web: www.wavecomp.ai}
    {Associate Technical Lead} {Oct 2018 - Jun 2019}
    {Project Lead} {Jun 2017 - Sep 2018}
    {Application Engineer} {Apr 2015 - May 2017}
        \item
        Lead the team which developed a Debug Architecture for the Data Processing Unit (DPU) of Wave Computing. Scrum was used in managing the project.
        \item
        Developed designs on Wave Computing’s Coarse Grain Re-configurable Array (CGRA) Architecture, compiled and simulated them on the complete SDK and tested on the chip. The designs include \textbf{Word2Vec} Machine Learning algorithm which was the company's flagship demo, Internally Mastered Direct Memory Access \textbf{(DMA) controller, Turbo LTE Decoder} and a Scalable Fixed Point Matrix Multiplication Acceleration Core.
        \item
        Development process included development of high-level models in C/C++, data-path design using low-level data-flow-graph language, test-bench generation, constraint based mapping to architecture, debugging the SDK by tracing generated concurrent instruction streams, development of highly optimized macros and memory templates using Wave’s assembly language and reporting improvements for the premature SDK and the architecture.
        \item
        Came up with an efficient methodology to place multiple compute kernels in a tightly packed manner on the CGRA and to route IOs and interconnects using different phase shifts which helped maximize resource utilization and gain a high throughput in the multi-threaded Word2Vec and Matrix Multiplication implementations.
	\end{rExperienceThreePos}

	\begin{rExperienceOnePos} {Fraunhofer Institute for Wind Energy Systems (IWES), Germany} {www.iwes.fraunhofer.de}
    {Intern} {Nov 2013 - May 2014}
        \item
        Conducted studies on the possibility to measure wind speed and other meteorological parameters using UAVs with inherent flight data (roll, pitch and yaw signals) and special sensors.
        \item
        Assisted in the development of a FPGA based crack detection system for mechanical structures of wind turbines with studies carried out on algorithms to process Acoustic Emission signals to identify crack initiation and propagation.
	\end{rExperienceOnePos}
\end{rSection}


%----------------------------------------------------------------------------------------
%	PROJECTS
%----------------------------------------------------------------------------------------

\begin{rSection}{Projects \& Courses}

    \begin{rLongProject}
    {Systematic Exploration of Optimization Potentials for the MGLET} {Pressure-Solver on Heterogeneous Hardware}
    {May 2023 - Present} {Master's Thesis (In progress)} 
        \item 
        Researching on ways to utilize Machine Learning methodologies to determine best solver preconditioner pairs for solving the Poisson-solver of MGLET implemented using PETSc on heterogeneous architectures based on properties of the matrix.
    \end{rLongProject}

    \begin{rLongProject}
    {Experimental Evaluation of Modern Computing Systems and} {Accelerators}
    {Apr 2023 - Jul 2023} {Master's Lab Course}
        \item
        Experiments carried out on the Bavarian Energy, Architecture and Software Test-bed (BEAST) of the Leibniz Supercomputing Centre (LRZ), Munich on CPU and GPU Architectures from Intel, AMD, NVidia, Marvell and Fujitsu.
        \item
        Experimented on Node-level performance optimizations, Cache optimizations, Multi-GPU offloading with OpenMP, SIMD programming, Profiling (using Perf, Likwid, Nsight-Systems), Scaling, Instruction-level parallelism, Branch prediction and NUMA behavior. 
    \end{rLongProject}

    \begin{rShortProject}
    {Parallelisation of Physics Calculations on GPUs with CUDA}
    {May 2021 - Jul 2021} {Master's Seminar}
        \item
        Implementation of parallelized 2D Discrete Fourier Transform and Fast Fourier Transform algorithms for double-slit experiment using CUDA and comparison of performance against cuFFT library implementations.
    \end{rShortProject}

\newpage

    \begin{rShortProject}
    {Computational Fluid Dynamics Lab}
    {Apr 2021 - Jul 2021} {Master's Lab Course}
        \item
        2D incompressible Navier-Stokes fluid solver implementation using modern C++ with parallelization using OpenMP and MPI.
        \item
        2D Lattice Boltzmann fluid solver implementation using C++ and parallelization with CUDA.
    \end{rShortProject}

    \begin{rShortProject}
    {Rapid Hearing Screening of Newborns}
    {May 2014 - Apr 2015} {Bachelor's thesis}
        \item
        Developed a hand-held device capable of identifying hearing impairments of neonates using the Automated Auditory Brain-stem Response (AABR). Investigated on improving the signal to noise ratio of the AABR signal using different signal processing algorithms and effective auditory stimulus delivery techniques. Findings were published at the Moratuwa Engineering Research Conference (MERCon) which got published in IEEE Xplore in 2016.
    \end{rShortProject}

    % \begin{rShortProject}
    % {Electrooculographic (EOG) artifact removal from EEG signals}{Dec 2014 - Apr 2015} {Bachelor's Seminar}
    %     \item
    %     Implementation of Independent Component Analysis and Wavelet Neural Network to remove EOG artifacts from EEG signals using MATLAB.
    % \end{rShortProject}

    % \begin{rShortProject}
    % {TROIKA framework for heart rate monitoring with PPG Signals}{Dec 2014 - Apr 2015} {Bachelor's Seminar}
    %     \item
    %     Implementation of the TROIKA framework that included Singular Spectrum Analysis, Sparse Signal Reconstruction and Spectral Peak Tracking of Photoplethysmographic (PPG) signals to measure the heart rate during physical exercise using data sets provided at IEEE Signal Processing Cup, 2015.
    % \end{rShortProject}

    % \begin{rCourses}
    % {Relevant Courses}
    %     \item
    %     Programming of Supercomputers,
    %     Parallel Programming,
    %     Advanced Programming,
    %     Advanced Computer Architecture,
    %     Distributed Systems,
    %     Computer Networks, 
    %     Digital Systems Design.
    % \end{rCourses}

    % \begin{rShortProject}
    % \end{rShortProject}

\end{rSection}

%----------------------------------------------------------------------------------------
%	RESEARCH
%----------------------------------------------------------------------------------------

%\begin{rSection}{Research}

	%Section content\ldots

%\end{rSection}

%----------------------------------------------------------------------------------------
%	PUBLICATIONS
%----------------------------------------------------------------------------------------

\begin{rSection}{Publications}

    \begin{rBulletedList}
    	\item 
        W. A. H. R. Weerathunge, D. M. S. L. Bandara, M. G. B. Amaratunga and A. C. De Silva, ``Robust algorithm for objective hearing screening of newborns using Automated Auditory Brain-stem Response", 2016 Moratuwa Engineering Research Conference (MERCon), Moratuwa, 2016, pp. 149-155.
    \end{rBulletedList}

\end{rSection}

%----------------------------------------------------------------------------------------
%	CERTIFICATIONS
%----------------------------------------------------------------------------------------

\begin{rSection}{Certifications}
    \begin{rCertifications}
    {Certified Scrum Master} {Jan 2019}
    {Scrum Alliance}
    \item Certificant ID: 000885047
    \item Certification Expiration:  23 January 2021
    \end{rCertifications}
\end{rSection}

%----------------------------------------------------------------------------------------
%	ACHIEVEMENTS
%----------------------------------------------------------------------------------------

\begin{rSection}{Honours \& Awards}
	\begin{rBulletedList}
        \item One of the five students who got selected from the class of 2019 ($\sim$50 students) of Computational Science and Engineering masters course at the Technical University of Munich for the Elite Graduate Program conducted by the Bavarian Graduate School of Computational Engineering (BGCE). 
        \item Ranked \textbf{3rd} out of nearly 200,000 candidates at the National Level in the G.C.E. Advanced Level Examination in 2009 in
        the Physical Science stream.
        \item Trophy and special prize awarded by Senior Research Scientist Dr. Sarath D. Gunapala at NASA for the most outstanding student in the Science stream in Nalanda College in 2009.
        % \item Merit Scholarship awarded by the Mahapola Higher Education Trust Fund of the Government of Sri Lanka (2010).
        % \item Dialog Merit Scholarship awarded by Dialog Axiata PLC, Sri Lanka (2010).
        % \item Excellence Scholarship awarded by People’s Bank, Sri Lanka (2010).
	\end{rBulletedList}
\end{rSection}

%----------------------------------------------------------------------------------------
%	LEADERSHIP AND TEAMWORK
%----------------------------------------------------------------------------------------

\begin{rSection}{Leadership}
	\begin{rShortProject}
    {Classical Music Society, University of Moratuwa} {Jun 2012 - Aug 2013}
    {President}
        \item Project Chairman of ``Yaathra 2012", university's biggest musical event, held in October 2012 at the Youth Centre, Maharagama to an audience of 1500, with the participation of nearly 150 students engaged in performing and organizing and with an expenditure of Rs. 1 million. This was the first time the event was held in such grand scale outside university premises.
    \end{rShortProject}
\end{rSection}

%----------------------------------------------------------------------------------------
%	EXTRA CURRICULAR ACTIVITIES
%----------------------------------------------------------------------------------------

% \begin{rSection}{Extra Curricular Activities}

% \end{rSection}


\end{document}
