%%%%%%%%%%%%%%%%%%%%%%%%%%%%%%%%%%%%%%%%%
% Medium Length Professional CV
% LaTeX Template
% Version 3.0 (December 17, 2022)
%
% This template originates from:
% https://www.LaTeXTemplates.com
%
% Author:
% Vel (vel@latextemplates.com)
%
% Original author:
% Trey Hunner (http://www.treyhunner.com/)
%
% License:
% CC BY-NC-SA 4.0 (https://creativecommons.org/licenses/by-nc-sa/4.0/)
%
%%%%%%%%%%%%%%%%%%%%%%%%%%%%%%%%%%%%%%%%%

%----------------------------------------------------------------------------------------
%	PACKAGES AND OTHER DOCUMENT CONFIGURATIONS
%----------------------------------------------------------------------------------------

\documentclass[
%a4paper, % Uncomment for A4 paper size (default is US letter)
11pt, % Default font size, can use 10pt, 11pt or 12pt
]{./assets/resume} % Use the resume class
% \usepackage{ebgaramond} % Use the EB Garamond font
\usepackage{helvet}

%----------------------------------------------------------------------------------------
%	NAME SECTION
%----------------------------------------------------------------------------------------

\name{Dimuthu Jayawardane} % Your name to appear at the top

\phone{+94773166850}
\linkedin{https://www.linkedin.com/in/dimuthuasiri/}
\email{sales@accelr.net}

% You can use the \address command up to 3 times for 3 different addresses or pieces of contact information
% Any new lines (\\) you use in the \address commands will be converted to symbols, so each address will appear as a single line.

% \address{Email \\ sales@acceler.net} % Email

% \address{WhatsApp \\ +94 (0)71 6487 689} % WhatsApp Number

% \address{Linkedin \\ https://www.linkedin.com/in/kavinga-upul-ekanayaka/} % LinkedIn Profile

%------------------------------------------------

\begin{document}

%----------------------------------------------------------------------------------------
%	TECHNICAL STRENGTHS SECTION
%----------------------------------------------------------------------------------------
\begin{rSection}{Technical Strengths}
	
	\def\arraystretch{1.5}
	
	\begin{tabular}{p{2.0in} p{4.5in}}
		\textbf{Expertise} & \emph{Static low power verification, Power aware simulation, RTL design, RTL simulation, Emulation, Prototyping, Debug} \\
		\textbf{Programming Languages} & \emph{Verilog, SystemVerilog, Python, C, C++} \\
		\textbf{Tools and Frameworks} & \emph{VC LP, Plato, VC, Verdi, Synplify, Xilinx Vivado} \\ 
		\textbf{Languages} & \emph{Sinhala-Native, English-Excellent} \\
	\end{tabular}
	
\end{rSection}

%----------------------------------------------------------------------------------------
%	EDUCATION SECTION
%----------------------------------------------------------------------------------------

\begin{rSection}{Education}
	
	\textbf{University of Moratuwa, Sri Lanka} \hfill \textit{2014 - 2018} \\ 
	B.Sc (Hons) in Electronic \& Telecommunication Engineering \\
	Status : First Class (GPA : 3.82 / 4.20)
	
\end{rSection}

%----------------------------------------------------------------------------------------
%	EXPERIENCE SECTION
%----------------------------------------------------------------------------------------

\begin{rSection}{Experience}
	%X-EPIC
	\begin{rSubsectionX}{X-EPIC}{www.x-epic.com}{Senior Product Engineer Level 1}{Dec 2022 - Jul 2024}
		\item Validated XEPIC simulation and emulation platform using four Open Source RISCV SoCs (OpenPiton, Hummingbirdv2 E203, Amber, Black-parrot) ensuring functionality and performance. 
		\item Developed C programs for multi-core OpenPiton RISC-V SoC (up to 64 cores) for testing parallel simulation feature.
		\item Developed Amber SoC as a Target Reference Design for XEPIC platform validation and created a complete document including all the RTL modifications and run steps. 
		\item Integrated Xilinx IPs (DDR4 Memory, UART) into OpenPiton design, enhancing communication and memory management. 
		\item Achieved successful Linux booting on Amber and OpenPiton SoCs in two modes by managing emulation and prototyping configurations. 
		\item Facilitated customer engagements and demonstrations, improving third-party IP integration using XEPIC HuaPro Emulation and Prototyping flow. 
		\item Developed AXI RAM for OpenPiton, enabling customer IP integration. Transitioned the customer engagement project to a cloud based verification platform (VLAB), showcasing the behavior of the XEPIC tools to customers and enhancing the remote verification capabilities. 
		\item Conducted memory testing for different memory synthesis directives (Distributed RAMs, Black RAMS, Ultra RAMs and Registers) to ensure the memory support in Emulation and Prototyping flow. 
		\item Verified advanced memory optimization options and memory merging techniques in XEPIC hardware flow, ensuring the functionality and performance of the system. Validated different XEPIC HuaPro memory partitioning strategies to optimize system performance and resource utilization. 
		\item Tested extensive support for third-party netlists in the HuaPro emulation and prototyping environment. This involved validating compatibility, optimizing integration processes, and ensuring smooth operation within the HuaPro system, thereby enhancing the versatility and utility of emulation/prototyping platforms.
		\item Performed critical power-aware simulation testing for key Unified Power Format (UPF) LRM commands, enhancing the accuracy and efficiency of XEPIC HDL simulator, Galaxsim. 
		\item Conducted knowledge sharing sessions for advanced low power verification techniques and IP Integration in Emulation/Prototyping flow.
	\end{rSubsectionX}
	%Synopsys
	\begin{rSubsectionX}{Synopsys}{www.synopsys.com}{Senior Application Engineer Level 1}{Feb 2018 - Nov 2022}
		\item Conducted Sign-off Abstract Model (SAM) validation to ensure compliance and performance of static low power designs.
		\item Validated more than 300 tool bugs and enhancements related to isolation. level shifters, power switches and retention to ensure advanced low power techniques such as power gating, retention, multi voltage design, and dynamic voltage scaling. 
		\item Provided critical support for field application engineers during release migrations for Apple, Intel, Socionext, Renesas and Toshiba. This included assisting with the transition to new software versions, ensuring smooth upgrades, and troubleshooting any issues that arose during the migration process. 
		\item Engaged in analyzing tool issues reported by customers. This involved close collaboration with development teams to implement fixes and enhancements, ensuring customer satisfaction and minimizing downtime. 
		\item Directly worked with on-site customer application engineers to understand customer specific needs and requirements. Provided tool enhancements and validated them. 
		\item Developed python script for analyzing tool release to release checker report differences. 
		\item This helped to faster the customer release migration and save time by around 30 percent. Automated VC LP release patch JIRAs follow up using a python script.
	\end{rSubsectionX}
	%Dialog
	\begin{rSubsectionX}{Dialog Axiata PLC}{www.dialog.lk}{Intern}{Aug 2016 - Dec 2016}
		\item Field exposure and study in the latest telecommunication technologies, overall planning and frequency planning strategies.
		\item Web development: Gaining exposure in MySQL, HTML, PHP, JavaScript
		\item Exposure in quality control tools, strategies and customer solutions
	\end{rSubsectionX}
	
\end{rSection}

%----------------------------------------------------------------------------------------
%	RESEARCH SECTION
%----------------------------------------------------------------------------------------
% \begin{rSection}{Research}
% 	% CEP 
% 	\textbf{Hardware Implementation of a Complex Event Processor} \hfill \textit{Apr 2012 - Apr 2013}\\
% 	A hardware accelerated complex event processor (CEP) platform was designed and implemented on FPGA with reference to WSO2 siddhi software CEP platform.
% 	The design is highly parameterized to enhance the flexibility, scalability and compatibility with the software platform.
% 	Achieved more than 10x performance than its software counterpart verified using a real world dataset.

	
% 	% cloud computing
% 	\textbf{Hardware Acceleration for Cloud computing architectures} \hfill \textit{Apr 2013 - Oct 2015}\\
% 	Thorough analysis of cloud computing architecture helped to find out Network virtualization as the major bottleneck.
% 	Parallel processing techniques were used to improve the QoS of network virtualization using a hardware switch fabric designed in FPGA

	
% \end{rSection}

%----------------------------------------------------------------------------------------
%	PUBLICATION SECTION
%----------------------------------------------------------------------------------------
% \begin{rSection}{Publications}

% 	% international conference
% 	\textbf{IEEE Conference Paper } \hfill  \textit{October 2014}\\
% 	%\begin{quote}
% 	Ekanayaka, K.U.B.; Pasqual, A., ``FPGA based custom accelerator architecture framework for complex event processing," \emph{TENCON 2014 - 2014 IEEE Region 10 Conference} , vol., no., pp.1,6, 22-25 Oct. 2014
% 	%\end{quote}

% \end{rSection}
 
%----------------------------------------------------------------------------------------
%	ACHIEVEMENTS SECTION
%----------------------------------------------------------------------------------------
% \begin{rSection}{Achievements}

% 	% ipho
% 	\textbf{International Physics Olympiad} \hfill \textit{Jul 2007}\\
% 	Member of the Sri Lanka team at IPHO, Isfahan, Iran.

% 	% Apho
% 	\textbf{Asian Physics Olympiad} \hfill \textit{Apr 2007}\\
% 	Member of the Sri Lanka team at APHO, Shanghai, China.
% 	\\
% 	\\
% 	\\

% \end{rSection}
%----------------------------------------------------------------------------------------
%	HOUNORS AND AWARDS SECTION
%---------------------------------------------------------------------------------------- 
\begin{rSection}{Honors and Awards}

	\textbf{Synopsys “Above and Beyond Award”} \hfill \textit{2018}\\
	Award for validating Sign-off Abstraction Model (SAM) flow for VC LP

	\textbf{OUSL IMPACTO, Robotics and Mobile App Challenge, Open University, Sri Lanka} \hfill \textit{2015}\\
	1st Place

	\textbf{STAT DAY, V.K Samaranayake Memorial Inter University Quiz, Colombo University, Sri Lanka} \hfill \textit{2015}\\
	1st Place

	\textbf{(G.C.E.) A/L examination in the Physical Science Stream} \hfill \textit{2012}\\
	10th Place (all-island ranking)

\end{rSection}

%----------------------------------------------------------------------------------------
%	PROJECTS SECTION
%----------------------------------------------------------------------------------------
\begin{rSection}{Projects}

	% FYP
	\textbf{Arsenic detection of ground water using Anodic Stripping Voltammetry} \hfill \textit{2018}\\
	Despite the availability of numerous methods to measure arsenic levels in groundwater within laboratory settings, creating a portable device posed a significant challenge. By collaborating with the Department of Chemistry at the University of Colombo, Sri Lanka, we successfully developed a portable water quality measurement unit. This unit utilizes the anodic stripping voltammetry method to measure arsenic levels in groundwater efficiently. 

	% FYP 
	\textbf{Processor Design and Implementation on FPGA Spartan 6 to down sample an image} \hfill \textit{2018}\\
	Developed a processor with custom ISA to read the pixel values of the image and down sample them and store back.  

\end{rSection}

%----------------------------------------------------------------------------------------
%	PROFFESSIONAL AFFILIATIONS SECTION
%----------------------------------------------------------------------------------------
\begin{rSection}{Professional Affiliations}
	% IEEE
	\textbf{Institute of Electrical and Electronics Engineers (IEEE)} \hfill \textit{Since 2018}\\
	Status : Member
	
\end{rSection}

%----------------------------------------------------------------------------------------
%	LEADERSHIP AND TEAMWORK SECTION
%----------------------------------------------------------------------------------------

\begin{rSection}{Leadership and Teamwork}

% DVCON
\textbf{President} \hfill \textit{2019} \\
Nature Circle of Synopsys Sri Lanka


% IEEE vTools
\textbf{Committee Member} \hfill \textit{2017} \\
Electronic Club, University of Moratuwa

% IEEE Region 10
\textbf{Batch Representative} \hfill \textit{2017} \\
Department of Electronic and Telecommunication Engineering, University of Moratuwa

\end{rSection}

\end{document}