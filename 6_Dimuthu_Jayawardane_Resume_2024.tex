%%%%%%%%%%%%%%%%%%%%%%%%%%%%%%%%%%%%%%%%%
% Medium Length Professional CV
% LaTeX Template
% Version 3.0 (December 17, 2022)
%
% This template originates from:
% https://www.LaTeXTemplates.com
%
% Author:
% Vel (vel@latextemplates.com)
%
% Original author:
% Trey Hunner (http://www.treyhunner.com/)
%
% License:
% CC BY-NC-SA 4.0 (https://creativecommons.org/licenses/by-nc-sa/4.0/)
%
%%%%%%%%%%%%%%%%%%%%%%%%%%%%%%%%%%%%%%%%%

%----------------------------------------------------------------------------------------
%	PACKAGES AND OTHER DOCUMENT CONFIGURATIONS
%----------------------------------------------------------------------------------------

\documentclass[
%a4paper, % Uncomment for A4 paper size (default is US letter)
11pt, % Default font size, can use 10pt, 11pt or 12pt
]{./assets/resume} % Use the resume class
% \usepackage{ebgaramond} % Use the EB Garamond font
\usepackage{helvet}

%----------------------------------------------------------------------------------------
%	NAME SECTION
%----------------------------------------------------------------------------------------

\name{Dimuthu Jayawardane} % Your name to appear at the top

\phone{+94773166850}
\linkedin{https://www.linkedin.com/in/dimuthuasiri/}
\email{sales@accelr.net}

% You can use the \address command up to 3 times for 3 different addresses or pieces of contact information
% Any new lines (\\) you use in the \address commands will be converted to symbols, so each address will appear as a single line.

% \address{Email \\ sales@acceler.net} % Email

% \address{WhatsApp \\ +94 (0)71 6487 689} % WhatsApp Number

% \address{Linkedin \\ https://www.linkedin.com/in/kavinga-upul-ekanayaka/} % LinkedIn Profile

%------------------------------------------------

\begin{document}

%----------------------------------------------------------------------------------------
%	TECHNICAL STRENGTHS SECTION
%----------------------------------------------------------------------------------------
\begin{rSection}{Technical Strengths}
	
	\def\arraystretch{1.5}
	
	\begin{tabular}{p{2.0in} p{4.5in}}
		\textbf{Expertise} & \emph{Static low power verification, Power aware simulation, RTL design, RTL simulation, Emulation, Prototyping, Debug} \\
		\textbf{Programming Languages} & \emph{Verilog, SystemVerilog, Python, C, C++} \\
		\textbf{Tools and Frameworks} & \emph{VC LP, Plato, VC, Verdi, Synplify, Xilinx Vivado} \\ 
		\textbf{Languages} & \emph{Sinhala-Native, English-Excellent} \\
	\end{tabular}
	
\end{rSection}

%----------------------------------------------------------------------------------------
%	EDUCATION SECTION
%----------------------------------------------------------------------------------------

\begin{rSection}{Education}
	
	\textbf{University of Moratuwa, Sri Lanka} \hfill \textit{2014 - 2018} \\ 
	B.Sc (Hons) in Electronic \& Telecommunication Engineering \\
	Status : First Class (GPA : 3.82 / 4.20)
	
\end{rSection}

%----------------------------------------------------------------------------------------
%	EXPERIENCE SECTION
%----------------------------------------------------------------------------------------

\begin{rSection}{Experience}
	%X-EPIC
	\begin{rSubsectionX}{X-EPIC}{www.x-epic.com}{Senior Product Engineer Level 1}{Dec 2022 - Jul 2024}
		\item RISCV SoC validation on simulation and emulation platforms
		\begin{list}{$\cdot$}{\leftmargin=1em}
			\setlength{\itemsep}{-0.5em} \vspace{-0.5em}
			\item Open Source RISCV: \textit{OpenPiton}, \textit{HummingBird}, \textit{Amber}, \textit{Blackparrot}: Brought these open-source RISC-V SoC designs on to X-EPIC platform and validated both simulation and emulation platforms, ensuring their functionality and performance. 
			\item Developed C programs to run on OpenPiton RISC-V multi-core (4 core and 64 core) SoC on X-EPIC simulation tools to test parallel simulation feature.
			\item Developed the \textit{Amber} SoC as a Target Reference Design to verify X-EPIC simulation and emulation platform (RTL simulation, synthesis, compilation, P\&R and hardware run) and document the RTL modifications and run steps.  
			\item Xilinx IP Integration for open-source SoC designs: Integrated Xilinx IPs such as DDR4 memory interface generator IP and UART IP into the \textit{OpenPiton} design, facilitating seamless communication and memory management. 
			\item Linux boot in emulation and prototyping modes: Successfully booted linux in both emulation and prototyping modes in \textit{Amber} and \textit{OpenPiton} SoC designs.
		\end{list}
		\item Remote customer engagements
		\begin{list}{$\cdot$}{\leftmargin=1em}
			\setlength{\itemsep}{-0.5em} \vspace{-0.5em}
			\item Customer demonstrations: demonstration of third-party IP integration into SoC design using XEPIC HuaPro emulation and prototyping flow utilizing the XEPIC DDR daughter board solutions to enhance customer engagement and satisfaction. 
			\item AXI Developments: Developed AXI RAM for \textit{OpenPiton}, completing AXI integration and bringing it to the emulation platform. This development facilitated customers in integrating their IPs into the X-EPIC provided \textit{OpenPiton} ecosystem.
			\item Cloud-based verification platform (VLAB): Transitioned the customer engagement project to a cloud-based verification platform (VLAB), showcasing the behavior of the X-EPIC tools to customers and enhancing X-EPIC's remote verification capabilities.
		\end{list}
		\item Product validation
		\begin{list}{$\cdot$}{\leftmargin=1em}
			\setlength{\itemsep}{-0.5em} \vspace{-0.5em}
			\item HuaPro/HuaEmu: Extended memory testing using open-source designs: Conducted memory testing for different memory synthesis directives (distributed RAMs, black RAMs, ultra RAMs and registers), advanced memory optimization options and memory merging techniques in X-EPIC hardware flow, ensuring the functionality and performance of the emulation systems.
			\item Memory partitioning: Validated different X-EPIC HuaPro memory partitioning strategies to optimize system performance and resource utilization.
			\item HuaPro emulation/prototyping support for third party net-lists: Provided extensive support for third-party net-lists in the HuaPro emulation and prototyping environment. This involved validating compatibility, optimizing integration processes, and ensuring smooth operation within the HuaPro system, thereby enhancing the versatility and utility of emulation/prototyping platforms.
			\item Galaxsim power aware simulation testing: Performed critical power-aware simulation testing for key (UPF) LRM commands, enhancing the accuracy and efficiency of XEPIC HDL simulator
		\end{list}
	\end{rSubsectionX}
	%Synopsys
	\begin{rSubsectionX}{Synopsys}{www.synopsys.com}{Senior Application Engineer }{Feb 2018 - Nov 2022}
		\item Over 4.5 years working experience in the static low power verification domain.
		\item VC LP experience: Conducted SAM validation to ensure compliance and performance of static low power designs. Validated more than 300 tool bugs and enhancements related to isolation. Level shifters, power switches and retention to ensure advance low power techniques such as power gating, retention, multi-voltage design, and dynamic voltage scaling.
		\item Remote Customer Engagements: Apple, Socionext, Renesas, Toshiba
		\item Developed python script for analyzing tool release to release QoR issues. This helped to faster the customer release migration and save time by around 20\%.
		\item Automated VC LP release patch JIRAs follow up using a python script. 
	\end{rSubsectionX}
	%Dialog
	\begin{rSubsectionX}{Dialog Axiata PLC}{www.dialog.lk}{Intern}{Aug 2016 - Dec 2016}
		\item Field exposure and study in the latest telecommunication technologies, overall planning and frequency planning strategies.
		\item Web development: Gaining exposure in MySQL, HTML, PHP, JavaScript
		\item Exposure in quality control tools, strategies and customer solutions
	\end{rSubsectionX}
	
\end{rSection}

%----------------------------------------------------------------------------------------
%	RESEARCH SECTION
%----------------------------------------------------------------------------------------
% \begin{rSection}{Research}
% 	% CEP 
% 	\textbf{Hardware Implementation of a Complex Event Processor} \hfill \textit{Apr 2012 - Apr 2013}\\
% 	A hardware accelerated complex event processor (CEP) platform was designed and implemented on FPGA with reference to WSO2 siddhi software CEP platform.
% 	The design is highly parameterized to enhance the flexibility, scalability and compatibility with the software platform.
% 	Achieved more than 10x performance than its software counterpart verified using a real world dataset.

	
% 	% cloud computing
% 	\textbf{Hardware Acceleration for Cloud computing architectures} \hfill \textit{Apr 2013 - Oct 2015}\\
% 	Thorough analysis of cloud computing architecture helped to find out Network virtualization as the major bottleneck.
% 	Parallel processing techniques were used to improve the QoS of network virtualization using a hardware switch fabric designed in FPGA

	
% \end{rSection}

%----------------------------------------------------------------------------------------
%	PUBLICATION SECTION
%----------------------------------------------------------------------------------------
% \begin{rSection}{Publications}

% 	% international conference
% 	\textbf{IEEE Conference Paper } \hfill  \textit{October 2014}\\
% 	%\begin{quote}
% 	Ekanayaka, K.U.B.; Pasqual, A., ``FPGA based custom accelerator architecture framework for complex event processing," \emph{TENCON 2014 - 2014 IEEE Region 10 Conference} , vol., no., pp.1,6, 22-25 Oct. 2014
% 	%\end{quote}

% \end{rSection}
 
%----------------------------------------------------------------------------------------
%	ACHIEVEMENTS SECTION
%----------------------------------------------------------------------------------------
\begin{rSection}{Achievements}

	% % ipho
	% \textbf{International Physics Olympiad} \hfill \textit{Jul 2007}\\
	% Member of the Sri Lanka team at IPHO, Isfahan, Iran.

	% % Apho
	% \textbf{Asian Physics Olympiad} \hfill \textit{Apr 2007}\\
	% Member of the Sri Lanka team at APHO, Shanghai, China.
	% \\
	% \\
	% \\

\end{rSection}
%----------------------------------------------------------------------------------------
%	HOUNORS AND AWARDS SECTION
%---------------------------------------------------------------------------------------- 
\begin{rSection}{Honors and Awards}

	\textbf{Synopsys “Above and Beyond Award”} \hfill \textit{2018}\\
	Award for validating Sign-off Abstraction Model (SAM) flow for VC LP

	\textbf{OUSL IMPACTO, Robotics and Mobile App Challenge, Open University, Sri Lanka} \hfill \textit{2015}\\
	1st Place

	\textbf{STAT DAY, V.K Samaranayake Memorial Inter University Quiz, Colombo University, Sri Lanka} \hfill \textit{2015}\\
	1st Place

	\textbf{(G.C.E.) A/L examination in the Physical Science Stream} \hfill \textit{2012}\\
	10th Place (all-island ranking)

\end{rSection}

%----------------------------------------------------------------------------------------
%	PROJECTS SECTION
%----------------------------------------------------------------------------------------
\begin{rSection}{Projects}

	% FYP
	\textbf{Arsenic detection of ground water using Anodic Stripping Voltammetry} \hfill \textit{2018}\\
	Despite the availability of numerous methods to measure arsenic levels in groundwater within laboratory settings, creating a portable device posed a significant challenge. By collaborating with the Department of Chemistry at the University of Colombo, Sri Lanka, we successfully developed a portable water quality measurement unit. This unit utilizes the anodic stripping voltammetry method to measure arsenic levels in groundwater efficiently. 

	% FYP 
	\textbf{Processor Design and Implementation on FPGA Spartan 6 to down sample an image} \hfill \textit{2018}\\
	Developed a processor with custom ISA to read the pixel values of the image and down sample them and store back.  

\end{rSection}

%----------------------------------------------------------------------------------------
%	PROFFESSIONAL AFFILIATIONS SECTION
%----------------------------------------------------------------------------------------
\begin{rSection}{Professional Affiliations}
	% IEEE
	\textbf{Institute of Electrical and Electronics Engineers (IEEE)} \hfill \textit{Since 2018}\\
	Status : Member
	
\end{rSection}

%----------------------------------------------------------------------------------------
%	LEADERSHIP AND TEAMWORK SECTION
%----------------------------------------------------------------------------------------

\begin{rSection}{Leadership and Teamwork}

% DVCON
\textbf{President} \hfill \textit{2019} \\
Nature Circle of Synopsys Sri Lanka


% IEEE vTools
\textbf{Committee Member} \hfill \textit{2017} \\
Electronic Club, University of Moratuwa

% IEEE Region 10
\textbf{Batch Representative} \hfill \textit{2017} \\
Department of Electronic and Telecommunication Engineering, University of Moratuwa

\end{rSection}

\end{document}