%%%%%%%%%%%%%%%%%%%%%%%%%%%%%%%%%%%%%%%%%
% Medium Length Professional CV
% LaTeX Template
% Version 3.0 (December 17, 2022)
%
% This template originates from:
% https://www.LaTeXTemplates.com
%
% Author:
% Vel (vel@latextemplates.com)
%
% Original author:
% Trey Hunner (http://www.treyhunner.com/)
%
% License:
% CC BY-NC-SA 4.0 (https://creativecommons.org/licenses/by-nc-sa/4.0/)
%
%%%%%%%%%%%%%%%%%%%%%%%%%%%%%%%%%%%%%%%%%

%----------------------------------------------------------------------------------------
%	PACKAGES AND OTHER DOCUMENT CONFIGURATIONS
%----------------------------------------------------------------------------------------

\documentclass[
	%a4paper, % Uncomment for A4 paper size (default is US letter)
	11pt, % Default font size, can use 10pt, 11pt or 12pt
]{./assets/resume} % Use the resume class

% \usepackage{ebgaramond} % Use the EB Garamond font
\usepackage{helvet}
\usepackage{color, soul}


%------------------------------------------------

\name{Dasun Pathirage} % Your name to appear at the top

\phone{+94773166850}
\linkedin{https://www.linkedin.com/in/dasun-pathirage/}
\email{sales@accelr.net}


% You can use the \address command up to 3 times for 3 different addresses or pieces of contact information
% Any new lines (\\) you use in the \address commands will be converted to symbols, so each address will appear as a single line.

\address{Email \\ sales@acceler.net} % Email

\address{WhatsApp \\ +94 (0)77 3166 850} % WhatsApp Number

\address{Linkedin \\ https://www.linkedin.com/in/dasun-pathirage/} % LinkedIn Profile

%----------------------------------------------------------------------------------------

\begin{document}


% \begin{tabularx}{\textwidth}{
% 	| >{\raggedright\arraybackslash}X 
% 	| >{\raggedleft\arraybackslash}X | }
% 	\hline
% 	{\huge\bf Kasun Buddhi} \\
% 	WhatsApp : Linkedin : Email
	
% 	& \raisebox{-\totalheight}{\includegraphics[width=0.3\textwidth]{logo.png}} \\
% 	\hline
% \end{tabularx} 

% \begin{tabularx}{\textwidth}{ |X|X| } 
% 	\hline
% 	cell3 & \multirow{3}{5cm}{Multiple row} \\ 
% 	cell6 &  \\ 
% 	cell9 &  \\ 
% 	\hline
% \end{tabularx}

% \begin{tabularx}{\textwidth}{
% 	 	 >{\raggedright\arraybackslash}X 
% 	 	 >{\raggedleft\arraybackslash}X  } 
% 	\smallskip
% 	{\huge\bf Kasun Buddhi} & 
% 	\multirow[c]{3}{*}{{\includegraphics[width=0.25\textwidth]{logo.png}}}\\ 
% 	WhatsApp : Linkedin : Email & \\
% \end{tabularx}

%----------------------------------------------------------------------------------------
%	TECHNICAL STRENGTHS SECTION
%----------------------------------------------------------------------------------------

\begin{rSection}{Technical Strengths}

	\def\arraystretch{1.5}

	\begin{tabular}{ l l}
		\textbf{Expertise} & \emph{RTL/Digital Design, Hardware Acceleration, Parallel Computing,} \\ 
		& \emph{ML \& Deep Learning, Cloud Computing} \\
		\textbf{Programming Languages} & \emph{C++, SystemVerilog, Python, Java, C\#} \\
		\textbf{Tools and Frameworks} & \emph{Xilinx Vivado, Lattice Radiant, OpenRoad} \\ 
		\textbf{Languages} & \emph{Sinhala-Native, English-Excellent} \\
	\end{tabular}

\end{rSection}

%----------------------------------------------------------------------------------------
%	EDUCATION SECTION
%----------------------------------------------------------------------------------------

\begin{rSection}{Education}

	\textbf{IIT, Sri Lanka} \hfill \textit{2021 - Present} \\ 
	M.Sc. in Advanced Software Engineering \\
	University of Westminster \\
	% Member of Eta Kappa Nu \\
	% Member of Upsilon Pi Epsilon \\
	% Overall GPA: 3.22/4.0 \\
	Status: Reading

	\textbf{Sri Lanka Technological Campus} \hfill \textit{2016 - 2020} \\ 
    B.Sc (Hons) in Electronic \& Telecommunication Engineering \\
	% \hl{Overall GPA: 2.9} \\
	% \hl{Status: Second Upper ??? }
	
	
\end{rSection}

%----------------------------------------------------------------------------------------
%	WORK EXPERIENCE SECTION
%----------------------------------------------------------------------------------------

\begin{rSection}{Experience}

	\begin{rSubsectionM}{ACCELR}{www.accelr.lk}{Senior Software Engineer}{Apr 2023 - Present}{Software Engineer}{Jan 2023 - Present}{}{}
        \item Mentor ACCELR team members on topics related to parallel computing and HPC.
        \item RTL Training: Mentor trainees in the RTL design and verification (SystemVerilog).
        \item RISC-V working-group: Actively involved in research tasks focusing on RISC-V architecture.
		\item Knowledge Sharing: Conducted knowledge-sharing sessions for University students on digital design with verilog to in order to foster better relationships with local Universities.
	\end{rSubsectionM}

	\begin{rSubsectionX}{Analog Inference}{www.analog-inference.com}{Engineering Consultant}{Dec 2020 - Present}
		\item Member of the Analog Inference back-end software development team
		\item Responsible for the development an mapping tool capable of automatically mapping a given neural network on to the Analog Inference data flow accelerator chip. Carried out experiments with new mapping policies and algorithms to optimize space utilization and performance.
		\item Managed the development, testing, and maintenance of the automated mapper tool and its test regression suite.
		\item Oversaw firmware updates and maintenance, ensuring robust and up-to-date system operations.
		\item Designed and implemented a model visualization tool tailored to the hardware-constrained mapping, to enhance understanding and troubleshooting capabilities and thereby simplify the debug process.
		\item Developed and maintained tools for generating detailed reports on utilization, frames per second (FPS), and power distribution, aiding in better resource management.
		\item Created an API for a Python interface to facilitate backend mapper interactions, enhancing its usability and accessibility.
		\item Conducted experiments and optimizations on various neural networks, including ResNet, YOLO, and FCN, driving significant improvements in model efficiency and accuracy.
		\item Maintained the software simulators for AI chips, enhancing development accuracy and reducing turnaround times.
		\item Performed performance analysis using ARM Development Studio for YOLOv5, identifying key areas for improvement in the host side software stack.
	\end{rSubsectionX}

	\begin{rSubsectionX}{Paraqum Technologies}{paraqum.com}{Intern}{Feb 2018 - Aug 2018}
		\item Developed a Slack messaging bot and an email verification system for Gmail using C++ with CURL, enhancing communication efficiencies.
		\item Automated SMS delivery using a GSM modem in C++, streamlining the messaging process.
		\item Engineered hardware accelerators for DDR memory read/write operations in Verilog and C++ (High-Level Synthesis), boosting system performance.
		\item Implemented hardware-accelerated AXI DMA IP in Verilog, enhancing data throughput and system reliability.
		\item Acquired hands-on experience in FPGA electronic design and parallel processing, contributing to a deeper understanding of network products and their applications.
	\end{rSubsectionX}

\end{rSection}

%----------------------------------------------------------------------------------------
%	PROJECTS
%----------------------------------------------------------------------------------------

\begin{rSection}{Projects}

	\textbf{Hardware Acceleration of OpenSSL Crypto Functions using FPGAs} \\
	This project focused on improving the SSL/TLS connection performance by accelerating parts of the AES and RSA algorithms to FPGA. The project utilized multi-prime RSA with four primes for th hardware implementation. While not focusing on random prime number generation, which can be externally computed and sent to the FPGA via the PCIe port, the implementation concentrated on RSA key generation and decryption. The approach used a constant public exponent value of 65537, consistent with OpenSSL standards, and employed different 'n' and 'd' private key values for each RSA channel.

	\textbf{Cloud base bank account management system} \\
	Built and deployed a bank account management system, AWS CDK, AWS S3, AWS Cloud front, CI/CD with AWS CodePipleine.

	\textbf{Java Multi-threaded Booking and Renting System for Tourists} \\
	Developed a robust Java multi-threaded application designed to handle high availability for booking and renting systems tailored for tourists. The project leveraged advanced monitoring concepts using the Runnable interface to ensure optimal performance and reliability. Deployed in the AWS Cloud, the system utilized a distributed architecture and microservices to ensure scalability and fault tolerance. This setup provided a seamless and efficient user experience, capable of handling concurrent requests and maintaining system integrity under heavy load.

\end{rSection}

%----------------------------------------------------------------------------------------
%	RESEARCH
%----------------------------------------------------------------------------------------

%\begin{rSection}{Research}

	%Section content\ldots

%\end{rSection}

%----------------------------------------------------------------------------------------
%	PUBLICATIONS
%----------------------------------------------------------------------------------------

\begin{rSection}{Publications}

	\normalfont{\textbf{D. Pathirage,}, H.P.D.K. Wijewardana, L.A.S. Lakshan, Hassaan Hydher, Lasith Yasakethu, "Multi-Prime RSA Verilog Implementation Using 4-Primes," \textit{2021 IEEE 30th International Conference on Information and Automation for Sustainability (ICIAfS)}, Colombo, Sri Lanka, 2021}(\href{https://ieeexplore.ieee.org/document/9605975}{paper})

\end{rSection}

%----------------------------------------------------------------------------------------
%	PROFESSIONAL AFFILIATIONS
%----------------------------------------------------------------------------------------

\begin{rSection}{Professional Affiliations}

	\textbf{IEEE Sri Lanka Chapter} \hfill \textit{Since 2018} \\ 
	Status : Associate Member
	% Membership No.: AM-30835

\end{rSection}

%----------------------------------------------------------------------------------------
%	ACHIEVEMENTS
%----------------------------------------------------------------------------------------

\begin{rSection}{Certificates \& Achievements}

	% \textbf{Undergraduate Contribution to the University Enhancement - SLTC RAIN} \hfill \textit{2019}

	\textbf{RISC-V Foundational Associate (RVFA) - Linux Foundation} \hfill \textit{Ongoing}

	\textbf{SystemVerilog for ASIC/FPGA Design \& Verification - University of Moratuwa} \hfill \textit{2023}

	\textbf{VSD - VLSI System Design (VSD-IAT)} \hfill \textit{2023}
	
	\textbf{Complete Modern C++ (C++11/14/17) - Udemy} \hfill \textit{2021}

	\textbf{Machine Learning - Stanford University - Coursera} \hfill \textit{2021}

	\textbf{Cryptography and Information Theory - University of Colorado - Coursera} \hfill \textit{2020}

	\textbf{Cloud Engineering with Google Cloud - Google Cloud - Coursera} \hfill \textit{2020}

	\textbf{NDG Linux Unhatched - Cisco Networking Academy} \hfill \textit{2020}

	\textbf{AWS Cloud Practitioner Essentials - Amazon Web Services (AWS)} \hfill \textit{2020}

	\textbf{FPGA Programming with VHDL - Pluralsight} \hfill \textit{2020}
	
	\textbf{SSL/TLS Protocol and Handshake Process - Udemy} \hfill \textit{2020}

\end{rSection}

%----------------------------------------------------------------------------------------
%	LEADERSHIP AND TEAMWORK
%----------------------------------------------------------------------------------------

\begin{rSection}{Leadership \& Teamwork}

	\textbf{Membership Development Subcommittee Chairman - IEEE SL} \hfill \textit{2020 - 2021}

	\textbf{Committee member - IEEE YPS SL} \hfill \textit{2020 - 2021}
	
	\textbf{Co-Chairman, IEEE SL SWY Congress} \hfill \textit{2019}

	\textbf{Founder President, IEEE Student Branch at SLTC} \hfill \textit{2018 - 2019}

	\textbf{Student Ambassador, IEEE PES HAC} \hfill \textit{2018 - 2019}

	\textbf{Student Ambassador, IEEE MadC} \hfill \textit{2019}

	\textbf{Advisor IES SLTC} \hfill \textit{2020}

	\textbf{Director, LEO Club of SLTC} \hfill \textit{2017 - 2018}

\end{rSection}

%----------------------------------------------------------------------------------------
%	EXTRA CURRICULAR ACTIVITIES
%----------------------------------------------------------------------------------------

% \begin{rSection}{Extra Curricular Activities}

% 	\textbf{Member of University Badminton team} \hfill \textit{2014 - 2018}

% 	\textbf{Member of University Taekwondo team} \hfill \textit{2014 - 2018}

% 	\textbf{Member of University Media club} \hfill \textit{2014 - 2018}

% \end{rSection}








%----------------------------------------------------------------------------------------
%	TECHNICAL STRENGTHS SECTION
%----------------------------------------------------------------------------------------

% \begin{rSection}{Technical Strengths}

% 	\begin{tabular}{@{} >{\bfseries}l @{\hspace{6ex}} l @{}}
% 		Computer Languages & Prolog, Haskell, AWK, Erlang, Scheme, ML \\
% 		Protocols \& APIs & XML, JSON, SOAP, REST \\
% 		Databases & MySQL, PostgreSQL, Microsoft SQL \\
% 		Tools & SVN, Vim, Emacs
% 	\end{tabular}

% \end{rSection}

%----------------------------------------------------------------------------------------
%	EXAMPLE SECTION
%----------------------------------------------------------------------------------------

%\begin{rSection}{Section Name}

	%Section content\ldots

%\end{rSection}

%----------------------------------------------------------------------------------------

\end{document}
