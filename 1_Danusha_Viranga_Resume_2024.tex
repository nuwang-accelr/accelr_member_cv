%%%%%%%%%%%%%%%%%%%%%%%%%%%%%%%%%%%%%%%%%
% Medium Length Professional CV
% LaTeX Template
% Version 3.0 (December 17, 2022)
%
% This template originates from:
% https://www.LaTeXTemplates.com
%
% Author:
% Vel (vel@latextemplates.com)
%
% Original author:
% Trey Hunner (http://www.treyhunner.com/)
%
% License:
% CC BY-NC-SA 4.0 (https://creativecommons.org/licenses/by-nc-sa/4.0/)
%
%%%%%%%%%%%%%%%%%%%%%%%%%%%%%%%%%%%%%%%%%

%----------------------------------------------------------------------------------------
%	PACKAGES AND OTHER DOCUMENT CONFIGURATIONS
%----------------------------------------------------------------------------------------

\documentclass[
%a4paper, % Uncomment for A4 paper size (default is US letter)
11pt, % Default font size, can use 10pt, 11pt or 12pt
]{./assets/resume} % Use the resume class
% \usepackage{ebgaramond} % Use the EB Garamond font
\usepackage{helvet}

%----------------------------------------------------------------------------------------
%	NAME SECTION
%----------------------------------------------------------------------------------------

\name{Danusha Viranga} % Your name to appear at the top

\phone{+94773166850}
\linkedin{https://www.linkedin.com/in/danusha-viranga-86607788/}
\email{sales@accelr.net}

% You can use the \address command up to 3 times for 3 different addresses or pieces of contact information
% Any new lines (\\) you use in the \address commands will be converted to symbols, so each address will appear as a single line.

% \address{Email \\ sales@acceler.net} % Email

% \address{WhatsApp \\ +94 (0)77 3166 850} % WhatsApp Number

% \address{Linkedin \\ https://www.linkedin.com/in/danusha-viranga-86607788/} % LinkedIn Profile

%------------------------------------------------

\begin{document}

%----------------------------------------------------------------------------------------
%	TECHNICAL STRENGTHS SECTION
%----------------------------------------------------------------------------------------
\begin{rSection}{Technical Strengths}
	
	\def\arraystretch{1.5}
	
	\begin{tabular}{p{2.0in} p{4.5in}}
		\textbf{Expertise} & \emph{RTL design, Testbench Verification, Simulation, Debugging} \\
		\textbf{Programming Languages} & \emph{Verilog, SystemVerilog, VHDL, Python, C++} \\
		\textbf{Tools and Frameworks} & \emph{VCS, Verdi, ModelSim} \\ 
		\textbf{Languages} & \emph{Sinhala-Native, English-Excellent} \\
	\end{tabular}
	
\end{rSection}

%----------------------------------------------------------------------------------------
%	EDUCATION SECTION
%----------------------------------------------------------------------------------------

\begin{rSection}{Education}
	
	\textbf{University of Ruhuna, Sri Lanka} \hfill \textit{2009 - Dec 2014} \\ 
	B.Sc (Hons) in Electrical \& Telecommunication Engineering \\
	% Status : Second Class Lower (GPA : x.xx / 4.20)
	
\end{rSection}

%----------------------------------------------------------------------------------------
%	EXPERIENCE SECTION
%----------------------------------------------------------------------------------------

\begin{rSection}{Experience}
	%X-EPIC
	\begin{rSubsectionX}{X-EPIC}{www.x-epic.com}{Senior Product Engineer Level 2}
		\item RTL Simulator Validation: Validated and tested the Galaxsim RTL simulator for the support of SystemVerilog, SV Constraints, SV Assertions, Coverage, GLS, UVM, etc.
		\item New feature planning for debugging capabilities and GUI: SV Constraints in Galaxsim.
		\item Parallel Simulator Testing: PV validation of parallel simulation of Galaxsim tool with design partitioning.
		\item GUI-Based RTL Debugger Testing: Tested the GUI-based RTL debugger FusionDebug tool with a diverse range of designs, enhancing its debug capabilities.
		\item SystemVerilog Interfaces Testing: Tested SystemVerilog programming interfaces such as DPI and PLI. 
		\item Customer Demonstration of Galaxsim RTL Simulation flow: Conducted comprehensive RTL design simulations using the Galaxsim tool and demonstrated the utilization of FusionDebug tool for effective debugging of simulation outputs.
		\item RISC-V SoC Validation on Simulation: Knowledgeable in RISC-V processor architecture and its applications in EDA tools and verification methodologies. Utilized RISC-V designs for parallel simulation feature validation.
	\end{rSubsectionX}
	%Synopsys
	\begin{rSubsectionX}{Synopsys}{www.synopsys.com}{Senior Application Engineer (PV)}{Aug 2015 - Nov 2022}
		\item Contributed as a product validation engineer for VCS constraints (one of components of VCS simulation tool). This involved project management (JIRA), test plan review, root cause analysis of customer issues and customer flow verification.
		\item Created comprehensive test strategies, test plans, automation scripts and executed them for the EDA tools and features and enhancements based on customer requirements.
		\item Contributed to maintaining quality of the VCS EDA product by signing-off on VCS releases after executing sanity testing, release checklist items, regression/benchmark convergence.
		\item Customer engagement: Contributed to on-site customer support at SAMSUNG Korea. Debugged issues and bugs in customer designs on VCS tool and reproduced the issues with smaller tests. Collaborated with R\&D team to fix bugs and increase the performance of VCS tool on customer designs during migration
	\end{rSubsectionX}
	%Atrenta
	\begin{rSubsectionX}{ATRENTA}{www.synopsys.com}{Verification Engineer (PV)}{Mar 2014 - Aug 2015}
		\item Executed central flow testing using in-house customer designs to identify performance issues, such as time and memory degradations, in the SpyGlass tool.
		\item Reported identified issues to the R\&D team and actively followed up to ensure timely resolution
	\end{rSubsectionX}
	
\end{rSection}

%----------------------------------------------------------------------------------------
%	RESEARCH SECTION
%----------------------------------------------------------------------------------------
% \begin{rSection}{Research}
% 	% CEP 
% 	\textbf{Hardware Implementation of a Complex Event Processor} \hfill \textit{Apr 2012 - Apr 2013}\\
% 	A hardware accelerated complex event processor (CEP) platform was designed and implemented on FPGA with reference to WSO2 siddhi software CEP platform.
% 	The design is highly parameterized to enhance the flexibility, scalability and compatibility with the software platform.
% 	Achieved more than 10x performance than its software counterpart verified using a real world dataset.

	
% 	% cloud computing
% 	\textbf{Hardware Acceleration for Cloud computing architectures} \hfill \textit{Apr 2013 - Oct 2015}\\
% 	Thorough analysis of cloud computing architecture helped to find out Network virtualization as the major bottleneck.
% 	Parallel processing techniques were used to improve the QoS of network virtualization using a hardware switch fabric designed in FPGA

	
% \end{rSection}

%----------------------------------------------------------------------------------------
%	PUBLICATION SECTION
%----------------------------------------------------------------------------------------
% \begin{rSection}{Publications}

% 	% international conference
% 	\textbf{IEEE Conference Paper } \hfill  \textit{October 2014}\\
% 	%\begin{quote}
% 	Ekanayaka, K.U.B.; Pasqual, A., ``FPGA based custom accelerator architecture framework for complex event processing," \emph{TENCON 2014 - 2014 IEEE Region 10 Conference} , vol., no., pp.1,6, 22-25 Oct. 2014
% 	%\end{quote}

% \end{rSection}
 
%----------------------------------------------------------------------------------------
%	ACHIEVEMENTS SECTION
%----------------------------------------------------------------------------------------
% \begin{rSection}{Achievements}

% 	% ipho
% 	\textbf{International Physics Olympiad} \hfill \textit{Jul 2007}\\
% 	Member of the Sri Lanka team at IPHO, Isfahan, Iran.

% 	% Apho
% 	\textbf{Asian Physics Olympiad} \hfill \textit{Apr 2007}\\
% 	Member of the Sri Lanka team at APHO, Shanghai, China.
% 	\\
% 	\\
% 	\\

% \end{rSection}
%----------------------------------------------------------------------------------------
%	HOUNORS AND AWARDS SECTION
%---------------------------------------------------------------------------------------- 
\begin{rSection}{Honors and Awards}

	\textbf{E W Karunaratne AWARD} \hfill \textit{2013}\\
	Best Undergraduate Project in Electrical Engineering offered by The Institution of Engineers, Sri Lanka (IESL)

\end{rSection}

%----------------------------------------------------------------------------------------
%	PROJECTS SECTION
%----------------------------------------------------------------------------------------
% \begin{rSection}{Projects}

% 	% Connect 6
% 	\textbf{FPGA implementation of Connect-6 game} \hfill \textit{2012}\\
% 	Hardware accelerated connect-6 game playing algorithm in spartan-6 FPGA platform using Verilog. 
% 	Competed at the International Conference on Field Programmable Technology 2012 - Seoul, South Korea.

% 	% FYP 
% 	\textbf{PC based open standard Radar display system} \hfill \textit{Mar 2011 - Dec 2011}\\
% 	Final Year Project of B.Sc. degree program.
% 	Developed an open standard Radar display system for AASL to use with a normal PC
% 	with Linux platform. Micro-C, Qt Designer and C++ were used as programming tools.
% 	Micro-controller based switching and tunneling unit was designed to acquire data coming from Radar towers and interface with software.

% \end{rSection}

%----------------------------------------------------------------------------------------
%	PROFFESSIONAL AFFILIATIONS SECTION
%----------------------------------------------------------------------------------------
\begin{rSection}{Professional Affiliations}
	% IESL
	\textbf{Institute of Engineers Sri Lanka (IESL)} \hfill \textit{Since 2014}\\
	Status : Member
	
\end{rSection}

%----------------------------------------------------------------------------------------
%	LEADERSHIP AND TEAMWORK SECTION
%----------------------------------------------------------------------------------------

% \begin{rSection}{Leadership and Teamwork}

% % DVCON
% \textbf{Sri Lanka Liason Chair} \hfill \textit{2022 - 2023} \\
% Design and Verification Conference (DVCon-India)


% % IEEE vTools
% \textbf{Member} \hfill \textit{2018} \\
% IEEE MGA vTools Committee


% % IEEE Region 10
% \textbf{Chairman} \hfill \textit{2015} \\
% IEEE Region 10 (Asia/Pacific) Congress - Colombo, Sri Lanka


% % IEEE R10 PAC
% \textbf{Member} \hfill \textit{2023} \\
% IEEE Region 10 (Asia/Pacific) Professional Activities Committee

% % IEEE sri lanka section
% \textbf{Assistant Treasurer} \hfill \textit{2015 - 2019}\\
% IEEE Sri Lanka section

% % Toastmasters
% \textbf{President} \hfill \textit{2015 - 2016} \\
% University of Moratuwa Toastmasters club

% % ICIAfS 
% \textbf{Publicity Chair} \hfill \textit{2014} \\
% 7\textsuperscript{th} IEEE International Conference on Information and Automation for Sustainability (ICIAfS)

% \end{rSection}

\end{document}