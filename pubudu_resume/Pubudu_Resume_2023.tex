%%%%%%%%%%%%%%%%%%%%%%%%%%%%%%%%%%%%%%%%%
% Medium Length Graduate Curriculum Vitae
% LaTeX Template
%
% This template has been downloaded from:
% http://www.latextemplates.com
%
% Original author:
% Rensselaer Polytechnic Institute (http://www.rpi.edu/dept/arc/training/latex/resumes/)
%
% Important note:
% This template requires the res.cls file to be in the same directory as the
% .tex file. The res.cls file provides the resume style used for structuring the
% document.
%
%%%%%%%%%%%%%%%%%%%%%%%%%%%%%%%%%%%%%%%%%

%----------------------------------------------------------------------------------------
%	PACKAGES AND OTHER DOCUMENT CONFIGURATIONS
%----------------------------------------------------------------------------------------

\documentclass[mm]{res} % Use the res.cls style 

\usepackage[margin={0.75in, 0.5in}]{geometry}

\usepackage{helvet} % Default font is the helvetica postscript font

%\usepackage{newcent} % To change the default font to the new century schoolbook postscript font uncomment this line and comment the one above

\setlength{\textwidth}{5.4in} % Text width of the document   

\begin{document}

%----------------------------------------------------------------------------------------
%	NAME SECTION
%----------------------------------------------------------------------------------------

\moveleft\hoffset\leftline{\huge\bf Pubudu Rathnayake} % Your name at the top
\emph{Linkedin}: https://www.linkedin.com/in/pubudur/ \\
\emph{GitHub}: https://github.com/PubuduR
 
\moveleft\hoffset\vbox{\hrule width\resumewidth height 1pt}\smallskip % Horizontal line after name; adjust line thickness by changing the '1pt'
 
%----------------------------------------------------------------------------------------

\begin{resume}

%----------------------------------------------------------------------------------------
%	CONTACT INFORMATION SECTION
%----------------------------------------------------------------------------------------
 
%\section{CONTACT INFORMATION}  

%\emph{Postal} : kadawalagedara, Thuththiripitigama, Via Kurunegala, Sri Lanka.\\ % Your address
%\emph{Mobile}: +94 71 648 7689 
%\emph{Email}: upul.ekanayaka@acceler.net\\
%\emph{Linkedin}: www.linkedin.com/in/kavinga-upul-ekanayaka-04575820/

%----------------------------------------------------------------------------------------
%	SKILLS SECTION
%----------------------------------------------------------------------------------------
\vspace{2 mm}
\section{SKILLS} 

\textbf{Skills and Expertise}
\begin{quote}
	\emph{ML \& Deep Learning, Data Analysing \& Visualization, Computer Vision, NLP, Algorithm \& Data Structures}
\end{quote}

\textbf{Computer Programming}
\begin{quote}
	\emph{Python, C/C++, JavaScript, Java, SQL}
\end{quote}

\textbf{Tools and Frameworks}
\begin{quote}
	\emph{Tensorflow, Pytorch, TensorRT, Onnx, Jupytor Notebook, Scikit-Learn, GCP-AWS, Docker, Jetson Nano, Git}
	\end{quote}

\textbf{Cloud Experience}
	\begin{quote}
		\emph{GCP, AWS, Azure}
		\end{quote}

\textbf{Operating Systems}
\begin{quote}
	\emph{Linux, Windows}
	\end{quote} 


%----------------------------------------------------------------------------------------
%	EDUCATION SECTION
%----------------------------------------------------------------------------------------

\section{EDUCATION}

% PhD information
%\textbf{Ph.D. candidate in Electronic Engineering} \hfill April 2012 - Present
%\begin{quote}
%(Expected in 2017 March)\\
%\emph{Department of Electronic and Telecommunication Engineering, \\
%University of Moratuwa, Sri Lanka.}
%\end{quote}

% BSc information
\textbf{B.Sc. Software Engineering (Honors)} \hfill 2018 - Nov 2021
\begin{quote}
\emph{University of Plymouth, UK Affiliated with NSBM Green University, Sri Lanka.}
\end{quote}

\begin{itemize} \itemsep -1pt % Reduce space between items
\item First Class
\item Final Mark : 71.43/100
\end{itemize}

% A/L information
%\textbf{G.C.E Advanced Level, Physical Science Stream} \hfill May 2006
%\begin{quote}
%\emph{Maliyadeva college, Kurunegala, Sri Lanka.}
%\end{quote}
%
%\begin{itemize} \itemsep -1pt % Reduce space between items
%\item Country Rank : 20
%\item District Rank : 03
%\item Z-score: 2.8793 with \textbf{A} grades in All 3 subjects.
%\end{itemize}

%----------------------------------------------------------------------------------------
%	PROFESSIONAL EXPERIENCE SECTION
%----------------------------------------------------------------------------------------

\section{WORKING EXPERIENCE}

%acceler
\textbf{Machine Learning Engineer} \hfill Aug 2022 - Present\\
\textbf{Trainee Machine Learning Enginee} \hfill Sep 2019 - Nov 2022
\begin{quote}
	\emph{ACCELR, Colombo, Sri Lanka \\
		web: www.accelr.lk}
\end{quote}
\begin{itemize} \itemsep -1pt 
\item Involved in architecting and implementing a FPGA based search engine accelerator as a POC for an early stage startup in USA
\item Utilizing and enhancing the Yahoo stream processor benchmark suit to compile performance numbers for  Apache Spark, Apache Storm and WSO2 stream processor
\item Involved in Acceler recruitment program
\end{itemize}
%bigstream
\textbf{FPGA design Engineer} \hfill Sep 2019 - Jan 2022
\begin{quote}
	\emph{Bigstream, Mountain view, CA, USA\\
		  web: www.bigstream.co}
\end{quote}
\begin{itemize} \itemsep -1pt 
	\item Held the delivery responsibility of Bigstream Sri Lanka team
	\item Deployed Bigstream openCL based FPGA accelerated platform (FAP) in AWS F1 platform
	\item Architected and implemented Bigstream FAP for Intel data center FPGAs
	\item Developed Rate controller reference RTL design and openCL application for testing and verifying Bigstream FAP using Xilinx Vitis and Intel Quartus design tools
	\item Performed a research based investigation of Bigstream FAP for spark benchmark performance in AWS F1 and Xilinx servers
	\item Developed Xilinx QDMA shell interface for Bigstream FAP
	\item Implemented multi-FPGA support for Bigstream FAP
	\item Worked with Systems engineering team for Jenkins based CI-CD pipelines
\end{itemize}
%wave
\textbf{Technical Lead} \hfill Oct 2018 - July 2019\\
\textbf{Team Lead} \hfill Oct 2016 - Sep 2018\\
\textbf{Application Engineer} \hfill Oct 2015 - Sep 2016
\begin{quote}
	\emph{Wave computing, Campbell, CA, USA\\
		  web: www.wavecomp.ai}
\end{quote}
\begin{itemize} \itemsep -1pt 
	\item Lead the Sri Lankan team in the implementation of machine learning inference algorithms on Wave DPU using generic building blocks
	\item Major contributor in the development of `word2vec' machine learning training algorithm on Wave DPU architecture
	\item word2vec was used as main reference design to test first wave hardware chip which had proven higher performance than CPU counterpart
	\item Implement the DMA controller for unit blocks in Wave DPU inference architecture
	\item Design and implement RTL level applications and C++ test benches, reference deigns to test the Wave SDK tool set (simulators/compilers)
	\item Drive front end simulator from application side with the use of Python scripting
	\item Provide feedback and strategic plans to SDK tool developers to improve front end and back end wave tools
	% \item Prepare reference applications to run on emulator/hardware after full functional testing in software
	\item Debug and report simulator/compiler issues to SDK developers
	% \item Master in wave SDK application regression suit	
	\item Involved in recruiting, training and mentoring team members and trainees
\end{itemize}

%research assistant
\textbf{Research Assistant} \hfill Apr 2012 - Oct 2015
\begin{quote}
	\emph{Department of Electronic and Telecommunication Engineering, \\
		University of Moratuwa, Sri Lanka.}
\end{quote}

%internship
% \textbf{Engineering Intern} \hfill Mar 2010 – Sep 2010
% \begin{quote}
% 	\emph{Airport and Aviation Services (Sri Lanka) Limited,\\
% 		Bandaranaike International Airport, Katunaike, Sri Lanka.}
% \end{quote}

%----------------------------------------------------------------------------------------
%	RESEARCH SECTION
%----------------------------------------------------------------------------------------
\vspace {2 mm}
\section{RESEARCH} 

% CEP 
\textbf{Hardware Implementation of a Complex Event Processor} \hfill Apr 2012 - Apr 2013
\begin{quote}
	\emph{Research Project}
\end{quote}

\begin{itemize} \itemsep -1pt % Reduce space between items
	\item A hardware accelerated complex event processor (CEP) platform was designed and implemented on FPGA with reference to WSO2 siddhi software CEP platform
	\item The design is highly parameterized to enhance the flexibility, scalability and compatibility with the software platform.
	\item Achieved more than 10x performance than its software counterpart verified using a real world dataset.
\end{itemize}

% cloud computing
\textbf{Hardware Acceleration for Cloud computing architectures} \hfill Apr 2013 - Oct 2015
\begin{quote}
\emph{Research Project}
\end{quote}

\begin{itemize} \itemsep -1pt % Reduce space between items
\item Thorough analysis of cloud computing architecture helped to find out Network virtualization as the major bottleneck
\item Parallel processing techniques were used to improve the QoS of network virtualization using a hardware switch fabric designed in FPGA
\end{itemize}

%----------------------------------------------------------------------------------------
%	PUBLICATION SECTION
%----------------------------------------------------------------------------------------
\vspace {2 mm}
\section{PUBLICATIONS} 

% international conference
\textbf{IEEE Conference Paper } \hfill October 2014
\begin{quote}
Ekanayaka, K.U.B.; Pasqual, A., ``FPGA based custom accelerator architecture framework for complex event processing," \emph{TENCON 2014 - 2014 IEEE Region 10 Conference} , vol., no., pp.1,6, 22-25 Oct. 2014
\end{quote}
 
%----------------------------------------------------------------------------------------
%	ACHIEVEMENTS SECTION
%----------------------------------------------------------------------------------------

\section{ACHIEVEMENTS} 

% ipho
\textbf{International Physics Olympiad} \hfill Jul 2007
\begin{quote}
\emph{Isfahan, Iran.}
\end{quote}

\begin{itemize} \itemsep -1pt % Reduce space between items
\item Member of the Sri Lankan Team.
\end{itemize}

% Apho
\textbf{Asian Physics Olympiad} \hfill Apr 2007
\begin{quote}
\emph{Shanghai, China.}
\end{quote}

\begin{itemize} \itemsep -1pt % Reduce space between items
\item Member of the Sri Lankan Team.
\end{itemize}

%----------------------------------------------------------------------------------------
%	INTERNATIONAL EXPOSURE SECTION
%----------------------------------------------------------------------------------------

% \section{INTERNATIONAL EXPOSURE} 

% % IEEE SC2014
% \textbf{IEEE Sections congress} \hfill Aug 2014
% \begin{quote}
% \emph{Amsterdam, Netherlands.}
% \end{quote}

% \begin{itemize} \itemsep -1pt % Reduce space between items
% \item Sri Lankan delegate.
% \end{itemize}

% % hp discover
% \textbf{HP Discover conference} \hfill Dec 2013
% \begin{quote}
% \emph{Barcelona, Spain.}
% \end{quote}

% \begin{itemize} \itemsep -1pt % Reduce space between items
% \item Sri Lankan delegate.
% \end{itemize}


%----------------------------------------------------------------------------------------
%	HOUNORS AND AWARDS SECTION
%---------------------------------------------------------------------------------------- 

\section{HOUNORS AND AWARDS}

%sri lankan physics olympiad
\textbf{Bronze Medal at Sri Lankan Physics Olympiad} \hfill Apr 2006
\begin{quote}
\emph{Institute of Physics, University of Colombo, Sri Lanka.}
\end{quote}

%----------------------------------------------------------------------------------------
%	PROJECTS SECTION
%----------------------------------------------------------------------------------------
\vspace{2 mm}
\section{PROJECTS} 

% Connect 6
\textbf{FPGA implementation of Connect-6 game} \hfill 2012
\begin{quote}
\emph{FPGA international competition}
\end{quote}

\begin{itemize} \itemsep -1pt % Reduce space between items
\item Hardware accelerated connect-6 game playing algorithm in spartan-6 FPGA platform using Verilog. 
\item Competed at the International Conference on Field Programmable Technology 2012 - Seoul, South Korea.
\end{itemize}


% FYP 
\textbf{Final Year Project} \hfill Mar 2011 - Dec 2011
\begin{quote}
\emph{PC based open standard Radar display system.}
\end{quote}

\begin{itemize} \itemsep -1pt % Reduce space between items
\item Developed an open standard Radar display system for AASL to use with a normal PC
with Linux platform. Micro-C, Qt Designer and C++ were used as programming tools.
\item Micro-controller based switching and tunneling unit was designed to acquire data coming from Radar towers and interface with software.
\end{itemize}

% DSD individual 
%\textbf{FPGA development of a convolutional encoder and a Viterbi decoder}\\
%\indent { \hspace{120 mm} Aug 2011} 
% \begin{quote}
% \emph{Course project}
% \end{quote}

% \begin{itemize} \itemsep -1pt % Reduce space between items
% \item Develop a Convolutional encoder and a Viterbi decoder using `verilog' as the HDL and implanted them on a vertex2pro FPGA board as an individual project.
% \end{itemize}

% DSD group 
%\textbf{FPGA development of synchronization of input data streams from
%different clock domains.} \hfill Aug 2011
% \begin{quote}
% \emph{Course project}
% \end{quote}

% \begin{itemize} \itemsep -1pt % Reduce space between items
% \item Develop synchronization of input data streams from different clock domains using verilog as the HDL and implemented it in vertex2pro FPGA development board.
% \end{itemize}

% SMPS 
%\textbf{Switch Mode Power Supply design} \hfill Jul 2011
%\begin{quote}
%\emph{Course project}
%\end{quote}

%\begin{itemize} \itemsep -1pt % Reduce space between items
%\item Design and develop a buck converter as the SMPS design and implemented it in a
%PCB.
%\end{itemize}

% AFTN Messenger
% \textbf{AFTN Messenger} \hfill Mar 2010 - May 2010
% \begin{quote}
% \emph{Project was done at AASL, when working as a trainee.}
% \end{quote}

% \begin{itemize} \itemsep -1pt % Reduce space between items
% \item A terminal messenger software for the windows platform developed with C\# in visual studio 2008 is currently used in the operation by the Airport and Aviation Services (Sri Lanka) Limited.
% \end{itemize}

% DSP project
% \textbf{ECG Beat Counter With Noise Rejection} \hfill Feb 2010
% \begin{quote}
% \emph{Course Project}
% \end{quote}

% \begin{itemize} \itemsep -1pt % Reduce space between items
% \item A Digital Signal Processing project using Matlab programming tool to reject power supply noise level and count the heart beat rate.
% \end{itemize}

% Intelligent game
% \textbf{Intelligent Game} \hfill Jan 2010
% \begin{quote}
% \emph{Course Project}
% \end{quote}

% \begin{itemize} \itemsep -1pt % Reduce space between items
% \item A computer game (an intelligent agent) with only single player mode using the AI algorithms developed using JAVA.
% \end{itemize}

% Multistage Amplifier
% \textbf{Multistage Amplifier Design} \hfill Dec 2009
% \begin{quote}
% \emph{Course Project}
% \end{quote}

% \begin{itemize} \itemsep -1pt % Reduce space between items
% \item A multistage amplifier with 100Hz-100 kHz bandwidth and 20dB to 60dB gain.
% \end{itemize}

% Active filter
% \textbf{Active Filter Design} \hfill Sep 2009
% \begin{quote}
% \emph{Course Project}
% \end{quote}

% \begin{itemize} \itemsep -1pt % Reduce space between items
% \item An active high pass filter with a cut off frequency 3 kHz which can be used to filter the harmonics band with minimal distortion to obtain a high quality audio output of an active audio cross over.
% \end{itemize}

% OOWrite text editor
% \textbf{OOWrite Text Editor} \hfill Apr 2009
% \begin{quote}
% \emph{Course Project}
% \end{quote}

% \begin{itemize} \itemsep -1pt % Reduce space between items
% \item A text editor which can edit text using various ways in command line format and in user interface format developed using JAVA programming language according to object oriented programming concepts.
% \end{itemize}

% Fire fighting robot
% \textbf{Robot Design and Competition 2008} \hfill Oct 2008
% \begin{quote}
% \emph{Course Project and Competition}
% \end{quote}

% \begin{itemize} \itemsep -1pt % Reduce space between items
% \item A mobile robot navigating avoiding the obstacles and capable of identifying and blowing out fire.
% \item Won the fifth place of the competition.
% \end{itemize}

% Fan regulator
% \textbf{Weather Controlled Fan Regulator} \hfill May 2008
% \begin{quote}
% \emph{Course (June Term) Project}
% \end{quote}

% \begin{itemize} \itemsep -1pt % Reduce space between items
% \item A power saving product which automatically controls the speed of a ceiling fan
% according to the temperature inside the room.
% \item Best Level 2 June term project in year 2008.
% \end{itemize}

%----------------------------------------------------------------------------------------
%	PROFFESSIONAL AFFILIATIONS SECTION
%----------------------------------------------------------------------------------------
\vspace{2 mm}
\section{PROFESSIONAL AFFILIATIONS} 

% IEEE
\textbf{Member of IEEE} \hfill Since 2009 

% IESL
% \textbf{Member of IESL} \hfill Since 2010

% Toastmasters
\textbf{Member of Toastmasters International} \hfill 2014 - 2016

%----------------------------------------------------------------------------------------
%	LEADERSHIP AND TEAMWORK SECTION
%----------------------------------------------------------------------------------------
\vspace{2 mm}
\section{LEADERSHIP AND TEAMWORK} 

% DVCON
\textbf{Sri Lanka Liason Chair} \hfill 2022 - 2023 
\begin{quote}
	\emph{Design and Verification Conference (DVCon-India)}
\end{quote}

% IEEE vTools
\textbf{Member} \hfill 2018 
\begin{quote}
\emph{IEEE MGA vTools Committee}
\end{quote}

% IEEE Region 10
\textbf{Chairman} \hfill 2015 
\begin{quote}
\emph{IEEE Region 10 (Asia/Pacific) Congress - Colombo, Sri Lanka}
\end{quote}

% IEEE R10 PAC
\textbf{Member} \hfill 2023 
\begin{quote}
	\emph{IEEE Region 10 (Asia/Pacific) Professional Activities Committee}
\end{quote}

% IEEE sri lanka section
\textbf{Assistant Treasurer} \hfill 2015 - 2019
\begin{quote}
\emph{IEEE Sri Lanka section}
\end{quote}

% IEEE sri lanka section
% \textbf{Educational Activities Chair} \hfill 2016
% \begin{quote}
% \emph{IEEE Sri Lanka section}
% \end{quote}

% Toastmasters
\textbf{President} \hfill 2015 - 2016 
\begin{quote}
\emph{University of Moratuwa Toastmasters club}
\end{quote}

% IEEE YP
% \textbf{Chairman} \hfill 2013 - 2014 
% \begin{quote}
% \emph{IEEE Young Professionals Sri Lanka}
% \end{quote}

% IEEE south asia
% \textbf{Sri Lankan Representative} \hfill 2013 - 2014 
% \begin{quote}
% \emph{IEEE Young Professionals South Asia cluster/Region 10}
% \end{quote}

% IEEE sri lanka section
%\textbf{Executive Committee Member} \hfill 2014 
%\begin{quote}
%\emph{IEEE Sri Lanka section}
%\end{quote}

% ICIAfS 
\textbf{Publicity Chair} \hfill 2014 
\begin{quote}
\emph{7\textsuperscript{th} IEEE International Conference on Information and Automation for Sustainability (ICIAfS 2014)}
\end{quote}

% IEEE day ambassador 
% \textbf{Section Ambassador} \hfill October 2014 
% \begin{quote}
% \emph{IEEE Day 2014, Sri Lanka section}
% \end{quote}

% Toastmasters
%\textbf{Treasurer} \hfill 2014 - 2015 
%\begin{quote}
%\emph{University of Moratuwa Toastmasters club}
%\end{quote}

% IEEE GOLD
%\textbf{Vice Chairman} \hfill 2012 - 2013 
%\begin{quote}
%\emph{IEEE GOLD Sri Lanka Affinity Group}
%\end{quote}

% Classical Music Society
% \textbf{Treasurer} \hfill May 2008 - Jun 2009 
% \begin{quote}
% \emph{Classical Music Society, University of Moratuwa.}
% \end{quote}

% IEEE uom
%\textbf{Finance committee Chair person} \hfill Aug 2009 - Feb 2010 
%\begin{quote}
%\emph{IEEE Student Branch, University of Moratuwa.}
%\end{quote} 

% uom cricket
% \textbf{Member of the Cricket team} \hfill Aug 2009 - Feb 2010 
% \begin{quote}
% \emph{University of Moratuwa, Sri Lanka.}
% \end{quote} 

% college cricket
%\textbf{Member of the Cricket team} \hfill 2001 - 2003 
%\begin{quote}
%\emph{Maliyadeva College, Kurunegala, Sri Lanka.}
%\end{quote} 

%----------------------------------------------------------------------------------------
%	REFEREES SECTION
%----------------------------------------------------------------------------------------
%\section{REFEREES}

% Dr.Suaris
%\textbf{Dr. Peter Suaris}\\
%\emph{B.Sc. Eng. (Moratuwa), M.Sc. (Kentucky), Ph.D. (Duke)}\\
%\emph{Distinguished Engineer at Analog Inference}\\
%\emph{Email: psuaris@yahoo.com}\\

% Dr. Esbensen
%\textbf{Dr. Henrik Esbensen}\\
%\emph{B.Sc. (Aarhus), M.Sc. (Michigan), Ph.D. (Aarhus)}\\
%\emph{CTO \& Co-Founder, DreamStart Labs }\\
%\emph{Email	: henrik@dreamstartlabs.com}\\

% Dr. Pasqual
% \textbf{Dr. Ajith Pasqual}\\
% \emph{B.Sc. Eng. (Moratuwa), M.Eng. (Tokyo), Ph.D. (Tokyo), MIEEE, MACM}\\
% \emph{Senior Lecturer- University of Moratuwa, Director/CEO - Paraqum Technologies }\\
% \emph{Email	: pasqual@ent.mrt.ac.lk}\\

\end{resume}
\end{document}