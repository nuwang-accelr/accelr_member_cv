%%%%%%%%%%%%%%%%%%%%%%%%%%%%%%%%%%%%%%%%%
% Medium Length Professional CV
% LaTeX Template
% Version 3.0 (December 17, 2022)
%
% This template originates from:
% https://www.LaTeXTemplates.com
%
% Author:
% Vel (vel@latextemplates.com)
%
% Original author:
% Trey Hunner (http://www.treyhunner.com/)
%
% License:
% CC BY-NC-SA 4.0 (https://creativecommons.org/licenses/by-nc-sa/4.0/)
%
%%%%%%%%%%%%%%%%%%%%%%%%%%%%%%%%%%%%%%%%%

%----------------------------------------------------------------------------------------
%	PACKAGES AND OTHER DOCUMENT CONFIGURATIONS
%----------------------------------------------------------------------------------------

\documentclass[
%a4paper, % Uncomment for A4 paper size (default is US letter)
11pt, % Default font size, can use 10pt, 11pt or 12pt
]{./assets/resume} % Use the resume class
% \usepackage{ebgaramond} % Use the EB Garamond font
\usepackage{helvet}

%----------------------------------------------------------------------------------------
%	NAME SECTION
%----------------------------------------------------------------------------------------

\name{Dasun Madhusanka} % Your name to appear at the top

\phone{+94773166850}
\linkedin{https://www.linkedin.com/in/dasun-madhusanka-964788220/}
\email{sales@accelr.net}

% You can use the \address command up to 3 times for 3 different addresses or pieces of contact information
% Any new lines (\\) you use in the \address commands will be converted to symbols, so each address will appear as a single line.

% \address{Email \\ sales@acceler.net} % Email

% \address{WhatsApp \\ +94 (0)71 6487 689} % WhatsApp Number

% \address{Linkedin \\ https://www.linkedin.com/in/kavinga-upul-ekanayaka/} % LinkedIn Profile

%------------------------------------------------

\begin{document}

%----------------------------------------------------------------------------------------
%	TECHNICAL STRENGTHS SECTION
%----------------------------------------------------------------------------------------
\begin{rSection}{Technical Strengths}
	
	\def\arraystretch{1.5}
	
	\begin{tabular}{p{2.0in} p{4.5in}}
		\textbf{Expertise} & \emph{RTL design, UVM, Simulation, Design verification, Formal Verification} \\
		\textbf{Programming Languages} & \emph{Verilog, SystemVerilog, VHDL, Python, C, Java, C\#} \\
		\textbf{Tools and Frameworks} & \emph{VCS, Verdi, Xilinx Vivado, Matlab} \\ 
		\textbf{Languages} & \emph{Sinhala-Native, English-Excellent} \\
	\end{tabular}
	
\end{rSection}

%----------------------------------------------------------------------------------------
%	EDUCATION SECTION
%----------------------------------------------------------------------------------------

\begin{rSection}{Education}
	
	\textbf{University of Moratuwa, Sri Lanka} \hfill \textit{2013 - 2018} \\ 
	B.Sc (Hons) in Electronic \& Telecommunication Engineering \\
	Status : First Class (GPA : 3.54 / 4.20)
	
\end{rSection}

%----------------------------------------------------------------------------------------
%	EXPERIENCE SECTION
%----------------------------------------------------------------------------------------

\begin{rSection}{Experience}
	%X-EPIC
	\begin{rSubsectionX}{X-EPIC}{www.x-epic.com}{Senior Product Engineer Level 1}{May 2022 - Jul 2024}
		\item Product validation of simulation, emulation and prototyping tools
		\item RTL simulator validation: Validated the RTL simulator Galaxsim according to the SystemVerilog LRM covering functional coverage, specify blocks, data types and other SystemVerilog constructs
		\item Formal verification tool validation: Validated formal verification tool GalaxFV with multiple large designs including RISC-V SoC \textit{OpenPiton} etc.
		\item Leading an internal study on CNN acceleration for a RISC-V based SoC platform for low power edge devices
		\item RISC-V SoC validation: Conducted extensive validation of the in-house tools using OpenPiton, RISC-V SoCs on simulation, parallel simulation, GLS (with SDF file generated from Vivado), Code coverage and functional coverage ensuring accurate functionality and performance
		\item Design bank implementation: Implemented a design bank for in-house quality of result and performance testing using open-source Verilog/SystemVerilog designs, facilitating improved testing and validation processes.
		\item SystemVerilog programming interface testing: Validate in-house simulator Galaxsim for System Verilog programming interfaces such as DPI, PLI.
		\item Validate script base debug tool: Validate script based RTL design and Waveform debug tool XPI covering Python, TCL and C++ with large number of RTL designs and Waveforms.
		\item GLS and SDF validation: Validate in-house simulator Galaxsim on GLS covering the SDF LRM, with unit level testing and large scale design testing by using large scale design and SDF files generated by Xilinx Vivado.
		\item Knowledge sharing sessions: Conducted knowledge-sharing sessions for peers related to simulation technologies, enhancing team expertise and collaborative learning.
	\end{rSubsectionX}
	%Synopsys
	\begin{rSubsectionX}{Synopsys}{www.synopsys.com}{Senior Application Engineer }{Feb 2018 - May 2022}
		\item Customer support Engineer: Working as a customer support engineer for both Synopsys RTL simulator VCS and debugger Verdi, worked with large number of Synopsys customers including on-site support for Samsung South Korea for number of occasions.
		\item Collaborated with over 300 customers and resolved over 1000 customer issus covering RTL design, product related issues, UVM, GLS, SDF, NLP, etc.
		\item Excellent experience with Synopsys tool list, specially with VCS and Verdi.
	\end{rSubsectionX}
	
\end{rSection}

%----------------------------------------------------------------------------------------
%	RESEARCH SECTION
%----------------------------------------------------------------------------------------
% \begin{rSection}{Research}
% 	% CEP 
% 	\textbf{Hardware Implementation of a Complex Event Processor} \hfill \textit{Apr 2012 - Apr 2013}\\
% 	A hardware accelerated complex event processor (CEP) platform was designed and implemented on FPGA with reference to WSO2 siddhi software CEP platform.
% 	The design is highly parameterized to enhance the flexibility, scalability and compatibility with the software platform.
% 	Achieved more than 10x performance than its software counterpart verified using a real world dataset.

	
% 	% cloud computing
% 	\textbf{Hardware Acceleration for Cloud computing architectures} \hfill \textit{Apr 2013 - Oct 2015}\\
% 	Thorough analysis of cloud computing architecture helped to find out Network virtualization as the major bottleneck.
% 	Parallel processing techniques were used to improve the QoS of network virtualization using a hardware switch fabric designed in FPGA

	
% \end{rSection}

%----------------------------------------------------------------------------------------
%	PUBLICATION SECTION
%----------------------------------------------------------------------------------------
% \begin{rSection}{Publications}

% 	% international conference
% 	\textbf{IEEE Conference Paper } \hfill  \textit{October 2014}\\
% 	%\begin{quote}
% 	Ekanayaka, K.U.B.; Pasqual, A., ``FPGA based custom accelerator architecture framework for complex event processing," \emph{TENCON 2014 - 2014 IEEE Region 10 Conference} , vol., no., pp.1,6, 22-25 Oct. 2014
% 	%\end{quote}

% \end{rSection}
 
%----------------------------------------------------------------------------------------
%	ACHIEVEMENTS SECTION
%----------------------------------------------------------------------------------------
% \begin{rSection}{Achievements}

% 	% ipho
% 	\textbf{International Physics Olympiad} \hfill \textit{Jul 2007}\\
% 	Member of the Sri Lanka team at IPHO, Isfahan, Iran.

% 	% Apho
% 	\textbf{Asian Physics Olympiad} \hfill \textit{Apr 2007}\\
% 	Member of the Sri Lanka team at APHO, Shanghai, China.
% 	\\
% 	\\
% 	\\

% \end{rSection}
%----------------------------------------------------------------------------------------
%	HOUNORS AND AWARDS SECTION
%---------------------------------------------------------------------------------------- 
% \begin{rSection}{Honors and Awards}

% 	%sri lankan physics olympiad
% 	\textbf{Bronze Medal at Sri Lankan Physics Olympiad} \hfill \textit{Apr 2006}\\
% 	Achieved a bronze medal at SLPHO organized by Institute of Physics, University of Colombo, Sri Lanka.


% \end{rSection}

%----------------------------------------------------------------------------------------
%	PROJECTS SECTION
%----------------------------------------------------------------------------------------
\begin{rSection}{Projects}

	\textbf{Wireless energy transfer device for body implant devices} \hfill \textit{2018}\\
	Design a wireless power transfer device for charge and provide power for medical, body implant devices such as pacemaker. Design this energy transfer device to work with most suitable magnetic frequency for human body. Also developed a communication method to communicate in between receiver and transmitter, And developed an ASIC design for the receiver to reduce the size.

	\textbf{RISC-V processor} \hfill \textit{2018}\\
	Developed an ASIC design for a processor based on the RISC-V architecture, and run a simple image processing algorithm.

	\textbf{Robot competition} \hfill \textit{2018}\\
	Built a robot to complete a given tasks such as line following, node and object detecting, grid solving (Calculating the shortest path), gripping a box.
	

\end{rSection}

%----------------------------------------------------------------------------------------
%	PROFFESSIONAL AFFILIATIONS SECTION
%----------------------------------------------------------------------------------------
\begin{rSection}{Professional Affiliations}
	% IEEE
	\textbf{Institute of Electrical and Electronics Engineers (IEEE)} \hfill \textit{Since 2018}\\
	Status : Member
	
\end{rSection}

%----------------------------------------------------------------------------------------
%	LEADERSHIP AND TEAMWORK SECTION
%----------------------------------------------------------------------------------------

% \begin{rSection}{Leadership and Teamwork}

% % DVCON
% \textbf{Sri Lanka Liason Chair} \hfill \textit{2022 - 2023} \\
% Design and Verification Conference (DVCon-India)


% % IEEE vTools
% \textbf{Member} \hfill \textit{2018} \\
% IEEE MGA vTools Committee


% % IEEE Region 10
% \textbf{Chairman} \hfill \textit{2015} \\
% IEEE Region 10 (Asia/Pacific) Congress - Colombo, Sri Lanka


% % IEEE R10 PAC
% \textbf{Member} \hfill \textit{2023} \\
% IEEE Region 10 (Asia/Pacific) Professional Activities Committee

% % IEEE sri lanka section
% \textbf{Assistant Treasurer} \hfill \textit{2015 - 2019}\\
% IEEE Sri Lanka section

% % Toastmasters
% \textbf{President} \hfill \textit{2015 - 2016} \\
% University of Moratuwa Toastmasters club

% % ICIAfS 
% \textbf{Publicity Chair} \hfill \textit{2014} \\
% 7\textsuperscript{th} IEEE International Conference on Information and Automation for Sustainability (ICIAfS)

% \end{rSection}

\end{document}