%%%%%%%%%%%%%%%%%%%%%%%%%%%%%%%%%%%%%%%%%
% Medium Length Professional CV
% LaTeX Template
% Version 3.0 (December 17, 2022)
%
% This template originates from:
% https://www.LaTeXTemplates.com
%
% Author:
% Vel (vel@latextemplates.com)
%
% Original author:
% Trey Hunner (http://www.treyhunner.com/)
%
% License:
% CC BY-NC-SA 4.0 (https://creativecommons.org/licenses/by-nc-sa/4.0/)
%
%%%%%%%%%%%%%%%%%%%%%%%%%%%%%%%%%%%%%%%%%

%----------------------------------------------------------------------------------------
%	PACKAGES AND OTHER DOCUMENT CONFIGURATIONS
%----------------------------------------------------------------------------------------

\documentclass[
%a4paper, % Uncomment for A4 paper size (default is US letter)
11pt, % Default font size, can use 10pt, 11pt or 12pt
]{./assets/resume} % Use the resume class
% \usepackage{ebgaramond} % Use the EB Garamond font
\usepackage{helvet}

%----------------------------------------------------------------------------------------
%	NAME SECTION
%----------------------------------------------------------------------------------------

\name{Aruna Premachandra} % Your name to appear at the top

\phone{+94773166850}
\linkedin{https://www.linkedin.com/in/aruna-premachandra-28ba0588/}
\email{sales@accelr.net}

% You can use the \address command up to 3 times for 3 different addresses or pieces of contact information
% Any new lines (\\) you use in the \address commands will be converted to symbols, so each address will appear as a single line.

% \address{Email \\ sales@acceler.net} % Email

% \address{WhatsApp \\ +94 (0)71 6487 689} % WhatsApp Number

% \address{Linkedin \\ https://www.linkedin.com/in/kavinga-upul-ekanayaka/} % LinkedIn Profile

%------------------------------------------------

\begin{document}

%----------------------------------------------------------------------------------------
%	TECHNICAL STRENGTHS SECTION
%----------------------------------------------------------------------------------------
\begin{rSection}{Technical Strengths}
	
	\def\arraystretch{1.5}
	
	\begin{tabular}{p{2.0in} p{4.5in}}
		\textbf{Expertise} & \emph{RTL design, functional/GUI validation, Static Verification, Simulation, Emulation, Low Power in ASIC} \\
		\textbf{Programming Languages} & \emph{Verilog, SystemVerilog, VHDL, UPF, Perl, Java, C} \\
		\textbf{Tools and Frameworks} & \emph{VC Static, VCS, Formality, Xilinx Vivado, Matlab} \\ 
		\textbf{Languages} & \emph{Sinhala-Native, English-Excellent} \\
	\end{tabular}
	
\end{rSection}

%----------------------------------------------------------------------------------------
%	EDUCATION SECTION
%----------------------------------------------------------------------------------------

\begin{rSection}{Education}
	
	\textbf{University of Peradeniya, Sri Lanka} \hfill \textit{Oct 2016} \\ 
	B.Sc (Hons) in Electrical \& Electronic Engineering \\
	% Status : First Class (GPA : 3.87 / 4.20)
	
\end{rSection}

%----------------------------------------------------------------------------------------
%	EXPERIENCE SECTION
%----------------------------------------------------------------------------------------

\begin{rSection}{Experience}
	%X-EPIC
	\begin{rSubsectionX}{X-EPIC}{www.x-epic.com}{Senior Product Engineer Level 2}{Oct 2023 - Jul 2024}
		\item Product Validation Owner of the Formal Verification Tool (Equivalency checker) GalaxEC.
		\item Conducted validation of other EDA tools: vSyn (logic synthesizer), HuaPro (prototyping System), GalaxSim/FusionDebug (simulation and debug tool)
		\item Conducted extensive validation of the Black Parrot SoC design using simulation and emulation platforms, ensuring functional correctness and robustness.
		\item Performed thorough validation of the OpenPiton SoC design, utilizing equivalence checkers to maintain consistency across different development stages.
		\item Developed a benchmark regression framework from the scratch, integrating multiple synthesizers and equivalence checkers to automate and enhance the design verification process.
	\end{rSubsectionX}
	%Synopsys
	\begin{rSubsectionX}{Synopsys}{www.synopsys.com}{Senior Application Engineer (PV)}{Nov 2016 - Oct 2023}
		\item Conducted validation for EDA verification tools: VC Static Low Power, VC Static Lint, Formality, VCS and Verdi.
		\item Machine Learning for Static Verification: Led the development and deployment of a machine learning feature for static verification, from specifications to customer delivery, enhancing debugging and root cause analysis.
		\item Signoff Abstract Model: Led the development and deployment of the "Signoff Abstract Model" feature, improving runtime efficiency at design signoff from Static Verification.
		\item Logic Optimization Techniques: Owner of delivering a feature for applying logic optimization techniques early in the SoC design process, suggesting design optimizations during Lint checking.
		\item Collaborated with client engineering teams globally for new feature introducing, onsite product deployments, testing and customer training. 
		\item Coordinated release management with multiple product teams across verification tools. 
		\item Maintained regression test suites and wrote automation scripts for effective bug detecting.
		\item Created and reviewed comprehensive test plans and test cases.
		\item Reviewed specifications and user guides to ensure customer requirements were met and to ensure industrial documentation guidelines are met.
	\end{rSubsectionX}
\end{rSection}

%----------------------------------------------------------------------------------------
%	RESEARCH SECTION
%----------------------------------------------------------------------------------------
% \begin{rSection}{Research}
% 	% CEP 
% 	\textbf{Hardware Implementation of a Complex Event Processor} \hfill \textit{Apr 2012 - Apr 2013}\\
% 	A hardware accelerated complex event processor (CEP) platform was designed and implemented on FPGA with reference to WSO2 siddhi software CEP platform.
% 	The design is highly parameterized to enhance the flexibility, scalability and compatibility with the software platform.
% 	Achieved more than 10x performance than its software counterpart verified using a real world dataset.

	
% 	% cloud computing
% 	\textbf{Hardware Acceleration for Cloud computing architectures} \hfill \textit{Apr 2013 - Oct 2015}\\
% 	Thorough analysis of cloud computing architecture helped to find out Network virtualization as the major bottleneck.
% 	Parallel processing techniques were used to improve the QoS of network virtualization using a hardware switch fabric designed in FPGA

	
% \end{rSection}

%----------------------------------------------------------------------------------------
%	PUBLICATION SECTION
%----------------------------------------------------------------------------------------
% \begin{rSection}{Publications}

% 	% international conference
% 	\textbf{IEEE Conference Paper } \hfill  \textit{October 2014}\\
% 	%\begin{quote}
% 	Ekanayaka, K.U.B.; Pasqual, A., ``FPGA based custom accelerator architecture framework for complex event processing," \emph{TENCON 2014 - 2014 IEEE Region 10 Conference} , vol., no., pp.1,6, 22-25 Oct. 2014
% 	%\end{quote}

% \end{rSection}
 
%----------------------------------------------------------------------------------------
%	ACHIEVEMENTS SECTION
%----------------------------------------------------------------------------------------
% \begin{rSection}{Achievements}

% 	% ipho
% 	\textbf{International Physics Olympiad} \hfill \textit{Jul 2007}\\
% 	Member of the Sri Lanka team at IPHO, Isfahan, Iran.

% 	% Apho
% 	\textbf{Asian Physics Olympiad} \hfill \textit{Apr 2007}\\
% 	Member of the Sri Lanka team at APHO, Shanghai, China.
% 	\\
% 	\\
% 	\\

% \end{rSection}
%----------------------------------------------------------------------------------------
%	HOUNORS AND AWARDS SECTION
%---------------------------------------------------------------------------------------- 
\begin{rSection}{Honors and Awards}

	%sri lankan physics olympiad
	\textbf{Synopsys “Above and Beyond Award”} \hfill \textit{2022}\\
	Exceptional performance in product validation


\end{rSection}

%----------------------------------------------------------------------------------------
%	PROJECTS SECTION
%----------------------------------------------------------------------------------------
\begin{rSection}{Projects}

	
	\textbf{Surge protecting system for set-top box and router} \hfill \textit{2016}\\
	Designed a surge protecting system for set-top box and router in Sri Lanka Telecom customer premises telephone network.

	\textbf{Remote monitoring system to monitor flow rate of a canal} \hfill \textit{2016}\\

	\textbf{Stage Light System using micro-controllers} \hfill \textit{2016}\\

\end{rSection}

%----------------------------------------------------------------------------------------
%	PROFFESSIONAL AFFILIATIONS SECTION
%----------------------------------------------------------------------------------------
\begin{rSection}{Professional Affiliations}
	% IESL
	\textbf{Institute of Engineers Sri Lanka (IESL)} \hfill \textit{Since 2016}\\
	Status : Associate Member
	
\end{rSection}

%----------------------------------------------------------------------------------------
%	LEADERSHIP AND TEAMWORK SECTION
%----------------------------------------------------------------------------------------

\begin{rSection}{Leadership and Teamwork}

\textbf{President} \hfill \textit{2016} \\
Electrical \& Electronic Engineering Society, University of Peradeniya.

\textbf{Organizing Committee} \hfill \textit{2016} \\
‘Arunella' social welfare project - Engineering Students Union, Faculty of Engineering, University of Peradeniya.

\end{rSection}

\end{document}