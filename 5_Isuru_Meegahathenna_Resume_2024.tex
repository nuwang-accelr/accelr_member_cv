%%%%%%%%%%%%%%%%%%%%%%%%%%%%%%%%%%%%%%%%%
% Medium Length Professional CV
% LaTeX Template
% Version 3.0 (December 17, 2022)
%
% This template originates from:
% https://www.LaTeXTemplates.com
%
% Author:
% Vel (vel@latextemplates.com)
%
% Original author:
% Trey Hunner (http://www.treyhunner.com/)
%
% License:
% CC BY-NC-SA 4.0 (https://creativecommons.org/licenses/by-nc-sa/4.0/)
%
%%%%%%%%%%%%%%%%%%%%%%%%%%%%%%%%%%%%%%%%%

%----------------------------------------------------------------------------------------
%	PACKAGES AND OTHER DOCUMENT CONFIGURATIONS
%----------------------------------------------------------------------------------------

\documentclass[
%a4paper, % Uncomment for A4 paper size (default is US letter)
11pt, % Default font size, can use 10pt, 11pt or 12pt
]{./assets/resume} % Use the resume class
% \usepackage{ebgaramond} % Use the EB Garamond font
\usepackage{helvet}

%----------------------------------------------------------------------------------------
%	NAME SECTION
%----------------------------------------------------------------------------------------

\name{Isuru Meegahathenna} % Your name to appear at the top

\phone{+94773166850}
\linkedin{https://www.linkedin.com/in/isuru-meegahathenna-5aa901228}
\email{sales@accelr.net}

% You can use the \address command up to 3 times for 3 different addresses or pieces of contact information
% Any new lines (\\) you use in the \address commands will be converted to symbols, so each address will appear as a single line.

% \address{Email \\ sales@acceler.net} % Email

% \address{WhatsApp \\ +94 (0)71 6487 689} % WhatsApp Number

% \address{Linkedin \\ https://www.linkedin.com/in/kavinga-upul-ekanayaka/} % LinkedIn Profile

%------------------------------------------------

\begin{document}

%----------------------------------------------------------------------------------------
%	TECHNICAL STRENGTHS SECTION
%----------------------------------------------------------------------------------------
\begin{rSection}{Technical Strengths}
	
	\def\arraystretch{1.5}
	
	\begin{tabular}{p{2.0in} p{4.5in}}
		\textbf{Expertise} & \emph{Verification IP, AMBA AXI, AHB, APB protocols, Simulation, RTL design, Static Verification} \\
		\textbf{Programming Languages} & \emph{Verilog, SystemVerilog, VHDL, Python, C, C++} \\
		\textbf{Tools and Frameworks} & \emph{UVM, Xilinx Vivado, Matlab} \\ 
		\textbf{Languages} & \emph{Sinhala-Native, English-Excellent} \\
	\end{tabular}
	
\end{rSection}

%----------------------------------------------------------------------------------------
%	EDUCATION SECTION
%----------------------------------------------------------------------------------------

\begin{rSection}{Education}
	
	\textbf{Birla Institute of Technology \& Science} \hfill \textit{2021 - 2023} \\ 
	M.Tech in Data Science and Engineering \\
	Status : Completed

	\textbf{University of Moratuwa, Sri Lanka} \hfill \textit{2012 - 2017} \\ 
	B.Sc (Hons) in Electronic \& Telecommunication Engineering \\
	Status : Second Class Upper Division (GPA : 3.64 / 4.20)
	
\end{rSection}

%----------------------------------------------------------------------------------------
%	EXPERIENCE SECTION
%----------------------------------------------------------------------------------------

\begin{rSection}{Experience}
	%X-EPIC
	\begin{rSubsectionX}{X-EPIC}{www.x-epic.com}{Senior Product Engineer Level 1}{JUn 2022 - Jul 2024}
		\item Designed and implemented SV UVM based APB VIP from scratch. Built the foundation for developing SV UVM based VIPs through this project.
		\item Contributed to development of AHB and AXI VIPs.
		\item Validated AXI and AHB VIPs using open-source RISC-V SoCs C910 and OpenPiton (Simulation with bare-metal).
		\item Evaluated a third party PCIe VIP.
		\item Delivered SoC solutions for two active customers. Provided integration solutions with \textit{OpenPiton} SoC in bare-metal and Linux boot environment.
		\item Developed and maintained RTL component IPs (AXI interconnect, AXI to APB converter, AHB RAM) for SoC solutions.
		\item Improved cross protocol solutions for \textit{OpenPiton} integrations (\textit{OpenPiton} NOC cache coherency protocol and AXI, AHB, APB protocols)
		\item Conducted knowledge-sharing sessions for peers in the areas of VIP, System Verilog UVM, AMBA protocols, PCIe protocol and \textit{OpenPiton} NOC cache coherency protocol, enhancing team expertise and collaborative learning.
	\end{rSubsectionX}
	%Synopsys
	\begin{rSubsectionM}{Synopsys}{www.synopsys.com}{Senior R\&D Engineer Level 1}{Apr 2019 - May 2022}{Application Engineer Level 2}{Apr 2017 - Apr 2019}{}{}
		\item Performed enhancements and bug fixes in Synopsys AMBA VIPs (AHB, APB, AXI, AXI stream) and sub-system level verification solutions utilizing simulation and debug tools (VCS and Verdi).
		\item Worked on custom features in AMBA VIPs requested by more than 5 customers, communicated with ARM IP team to ensure the protocol compatibility.
		\item Implemented performance matrices and enhanced performance of AHB VIP.
		\item Developed latest protocol features for APB and AHB VIPs.
		\item Ownerships: Worked as the R\&D owner for the AHB, APB and AXI4 stream VIPs. Responsible for managing the quality of these VIPs, handling customer requirements and maintaining the internal regression suite for these VIPs.
		\item Conducted product validation and test planning on Synopsys static verification tools (Spyglass, VC Static).
		\item Analyzed requirements for feature requests from more than 3 customers.
		\item Validated different features in VC Static compiler like VHDL-2008 LRM support, constant propagation and memory model handling.
		\item Collaborated closely with R\&D engineers, CDC, RDC and LP PV teams.
	\end{rSubsectionM}
	%wavecomputing
	\begin{rSubsectionX}{Wave Computing, US}{www.wavecomp.ai}{Intern}{Oct 2015 - Apr 2016}
		\item Completed 6 months internship at Wave Computing (hosted by ParaQum Technologies (Pvt) Ltd in Sri Lanka).
		\item Involved in software tool chain development for ongoing projects (C++).
		\item Created a software bloom filter in python.
	\end{rSubsectionX}
	
\end{rSection}

%----------------------------------------------------------------------------------------
%	RESEARCH SECTION
%----------------------------------------------------------------------------------------
% \begin{rSection}{Research}
% 	% CEP 
% 	\textbf{Hardware Implementation of a Complex Event Processor} \hfill \textit{Apr 2012 - Apr 2013}\\
% 	A hardware accelerated complex event processor (CEP) platform was designed and implemented on FPGA with reference to WSO2 siddhi software CEP platform.
% 	The design is highly parameterized to enhance the flexibility, scalability and compatibility with the software platform.
% 	Achieved more than 10x performance than its software counterpart verified using a real world dataset.

	
% 	% cloud computing
% 	\textbf{Hardware Acceleration for Cloud computing architectures} \hfill \textit{Apr 2013 - Oct 2015}\\
% 	Thorough analysis of cloud computing architecture helped to find out Network virtualization as the major bottleneck.
% 	Parallel processing techniques were used to improve the QoS of network virtualization using a hardware switch fabric designed in FPGA

	
% \end{rSection}

%----------------------------------------------------------------------------------------
%	PUBLICATION SECTION
%----------------------------------------------------------------------------------------
% \begin{rSection}{Publications}

% 	% international conference
% 	\textbf{IEEE Conference Paper } \hfill  \textit{October 2014}\\
% 	%\begin{quote}
% 	Ekanayaka, K.U.B.; Pasqual, A., ``FPGA based custom accelerator architecture framework for complex event processing," \emph{TENCON 2014 - 2014 IEEE Region 10 Conference} , vol., no., pp.1,6, 22-25 Oct. 2014
% 	%\end{quote}

% \end{rSection}
 
%----------------------------------------------------------------------------------------
%	ACHIEVEMENTS SECTION
%----------------------------------------------------------------------------------------
% \begin{rSection}{Achievements}

% 	% ipho
% 	\textbf{International Physics Olympiad} \hfill \textit{Jul 2007}\\
% 	Member of the Sri Lanka team at IPHO, Isfahan, Iran.

% 	% Apho
% 	\textbf{Asian Physics Olympiad} \hfill \textit{Apr 2007}\\
% 	Member of the Sri Lanka team at APHO, Shanghai, China.
% 	\\
% 	\\
% 	\\

% \end{rSection}
%----------------------------------------------------------------------------------------
%	HOUNORS AND AWARDS SECTION
%---------------------------------------------------------------------------------------- 
% \begin{rSection}{Honors and Awards}

% 	%sri lankan physics olympiad
% 	\textbf{Bronze Medal at Sri Lankan Physics Olympiad} \hfill \textit{Apr 2006}\\
% 	Achieved a bronze medal at SLPHO organized by Institute of Physics, University of Colombo, Sri Lanka.


% \end{rSection}

%----------------------------------------------------------------------------------------
%	PROJECTS SECTION
%----------------------------------------------------------------------------------------
\begin{rSection}{Projects}

	% FYP
	\textbf{Smart insole for gait analysis} \hfill \textit{2017}\\
	Development of a pair of insoles which is capable of fall detection and fall prediction, easily worn by the wearer and does not hinder his daily activities. Used pressure sensors, IMUs, GPS trackers and BLE modules. Android application for self-monitoring.

	\textbf{Implementing a microprocessor in Verilog} \hfill \textit{2017}

	\textbf{Implementing a signal generator using an FPGA} \hfill \textit{2017}

	\textbf{Election results management system} \hfill \textit{2017}\\
	A system written in Java that enables the election officers to enter results which simultaneously updates results in the election commissioner’s office.

	\textbf{Advanced line following and grid solving robot and manual robot} \hfill \textit{2014}
	Autonomous and manual robots developed for SLRC 2014


\end{rSection}

%----------------------------------------------------------------------------------------
%	PROFFESSIONAL AFFILIATIONS SECTION
%----------------------------------------------------------------------------------------
\begin{rSection}{Professional Affiliations}
	% IESL
	\textbf{Institute of Electrical and Electronics Engineers (IEEE)} \hfill \textit{Since 2017}\\
	Status : Member
	
\end{rSection}

%----------------------------------------------------------------------------------------
%	LEADERSHIP AND TEAMWORK SECTION
%----------------------------------------------------------------------------------------

% \begin{rSection}{Leadership and Teamwork}

% % DVCON
% \textbf{Sri Lanka Liason Chair} \hfill \textit{2022 - 2023} \\
% Design and Verification Conference (DVCon-India)


% % IEEE vTools
% \textbf{Member} \hfill \textit{2018} \\
% IEEE MGA vTools Committee


% % IEEE Region 10
% \textbf{Chairman} \hfill \textit{2015} \\
% IEEE Region 10 (Asia/Pacific) Congress - Colombo, Sri Lanka


% % IEEE R10 PAC
% \textbf{Member} \hfill \textit{2023} \\
% IEEE Region 10 (Asia/Pacific) Professional Activities Committee

% % IEEE sri lanka section
% \textbf{Assistant Treasurer} \hfill \textit{2015 - 2019}\\
% IEEE Sri Lanka section

% % Toastmasters
% \textbf{President} \hfill \textit{2015 - 2016} \\
% University of Moratuwa Toastmasters club

% % ICIAfS 
% \textbf{Publicity Chair} \hfill \textit{2014} \\
% 7\textsuperscript{th} IEEE International Conference on Information and Automation for Sustainability (ICIAfS)

% \end{rSection}

\end{document}