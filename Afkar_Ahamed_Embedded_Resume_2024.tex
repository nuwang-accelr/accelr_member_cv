%%%%%%%%%%%%%%%%%%%%%%%%%%%%%%%%%%%%%%%%%
% Medium Length Professional CV
% LaTeX Template
% Version 3.0 (December 17, 2022)
%
% This template originates from:
% https://www.LaTeXTemplates.com
%
% Author:
% Vel (vel@latextemplates.com)
%
% Original author:
% Trey Hunner (http://www.treyhunner.com/)
%
% License:
% CC BY-NC-SA 4.0 (https://creativecommons.org/licenses/by-nc-sa/4.0/)
%
%%%%%%%%%%%%%%%%%%%%%%%%%%%%%%%%%%%%%%%%%

%----------------------------------------------------------------------------------------
%	PACKAGES AND OTHER DOCUMENT CONFIGURATIONS
%----------------------------------------------------------------------------------------

\documentclass[
	%a4paper, % Uncomment for A4 paper size (default is US letter)
	11pt, % Default font size, can use 10pt, 11pt or 12pt
]{./assets/resume} % Use the resume class

% \usepackage{ebgaramond} % Use the EB Garamond font
\usepackage{helvet}


%------------------------------------------------

\name{Afkar Ahamed} % Your name to appear at the top

\phone{+94773166850}
\linkedin{https://www.linkedin.com/in/afkar-ahamed/}
\email{sales@accelr.net}


% You can use the \address command up to 3 times for 3 different addresses or pieces of contact information
% Any new lines (\\) you use in the \address commands will be converted to symbols, so each address will appear as a single line.

\address{Email \\ sales@acceler.net} % Email

\address{WhatsApp \\ +94 (0)77 3166 850} % WhatsApp Number

\address{Linkedin \\ https://www.linkedin.com/in/afkar-ahamed/} % LinkedIn Profile

%----------------------------------------------------------------------------------------

\begin{document}


% \begin{tabularx}{\textwidth}{
% 	| >{\raggedright\arraybackslash}X 
% 	| >{\raggedleft\arraybackslash}X | }
% 	\hline
% 	{\huge\bf Kasun Buddhi} \\
% 	WhatsApp : Linkedin : Email
	
% 	& \raisebox{-\totalheight}{\includegraphics[width=0.3\textwidth]{logo.png}} \\
% 	\hline
% \end{tabularx} 

% \begin{tabularx}{\textwidth}{ |X|X| } 
% 	\hline
% 	cell3 & \multirow{3}{5cm}{Multiple row} \\ 
% 	cell6 &  \\ 
% 	cell9 &  \\ 
% 	\hline
% \end{tabularx}

% \begin{tabularx}{\textwidth}{
% 	 	 >{\raggedright\arraybackslash}X 
% 	 	 >{\raggedleft\arraybackslash}X  } 
% 	\smallskip
% 	{\huge\bf Kasun Buddhi} & 
% 	\multirow[c]{3}{*}{{\includegraphics[width=0.25\textwidth]{logo.png}}}\\ 
% 	WhatsApp : Linkedin : Email & \\
% \end{tabularx}

%----------------------------------------------------------------------------------------
%	TECHNICAL STRENGTHS SECTION
%----------------------------------------------------------------------------------------

\begin{rSection}{Technical Strengths}

	\def\arraystretch{1.5}

	\begin{tabular}{ l l}
		\textbf{Expertise} & \emph{Parallel Computing, RTL/Digital Design, Design Verification} \\
		\textbf{Programming Languages} & \emph{C, C++, Python, SystemVerilog, Java} \\
		\textbf{Microcontrollers/Microprocessors} & \emph{PIC, MSP430, ATMEGA, Raspbery Pi, ESP Wroom} \\
		\textbf{Embedded Tools and Frameworks} & \emph{ESP-IDF, EasyEDA, Proteus, Multisim} \\ 
		\textbf{Other Tools and Frameworks} & \emph{UVM, Xilinx Vivado, Altera Quartus, MPI, OpenMP, Pthreads} \\ 
		\textbf{Languages} & \emph{Tamil-Native, Sinhala-Excellent, English-Excellent} \\
	\end{tabular}

\end{rSection}

%----------------------------------------------------------------------------------------
%	EDUCATION SECTION
%----------------------------------------------------------------------------------------

\begin{rSection}{Education}

	\textbf{Sri Lanka Institute of Information Technology} \hfill \textit{2020 - 2023} \\ 
	B.S (Hons) Electrical and Electronic Engineering \\
	Overall GPA: 3.67/4 \\
	Status: Second Upper

	\textbf{Chartered Institute of Management Accountants (CIMA)} \hfill \textit{2023 - Present} \\ 
	Foundation Level \\
	% Overall GPA: 3.67/4 \\
	Status: Currently Reading
	
\end{rSection}

%----------------------------------------------------------------------------------------
%	WORK EXPERIENCE SECTION
%----------------------------------------------------------------------------------------

\begin{rSection}{Experience}

	\begin{rSubsectionX}{ACCELR}{www.accelr.lk}{Electronic Engineer}{Feb 2024 - Present}
		\item Contributing to UVM testbench developed tasks.
	\end{rSubsectionX}

	\begin{rSubsectionX}{ACCELR}{www.accelr.lk}{Intern - Electronic Engineer}{Nov 2022 - Jan 2023}
		\item Worked as a trainee design verification engineer.
		\item Implemented a UVM testbench for an ALU.
	\end{rSubsectionX}

	\begin{rSubsectionX}{Sanken Constructions (Pvt) Ltd}{sankenconstruction.com}{Intern - Electrical Engineer}{Nov 2021 - Jan 2022}
		\item Was assigned electrical engineering tasks such as installation and maintenance of transformers
		\item Was assigned electronics related projects such as networking, programming and troubleshooting of RCUs (room control unit).
	\end{rSubsectionX}

\end{rSection}

%----------------------------------------------------------------------------------------
%	PROJECTS
%----------------------------------------------------------------------------------------

\begin{rSection}{Projects}

	\textbf{A GPU accelerated inference framework for CNNs on Raspberry Pi 4} \\
	As a key member of a team developing a GPU-accelerated machine learning inference framework for the Raspberry Pi, I was responsible for designing and implementing high-performance compute kernels in C++. My role involved optimizing these kernels for GPU operations, significantly enhancing the efficiency of the framework in machine learning tasks like tensor computations and neural network processes. Additionally, I conducted comprehensive benchmark tests comparing GPU and CPU performance, ensuring optimal effectiveness of our solutions.

	\textbf{Object detection helmet for visually impaired persons} \\
	I led a team of four students in designing an object detection helmet for visually impaired persons which got selected for the final rounds in a competition conducted by IEEE students branch of Sri Lanka. My primary contribution in the project was to lead the team and handling sensors and actuators.

	\textbf{Smart Irrigation System} \\
	Designed and implemented an embedded system which was able to control the irrigation, based on water level, and other environmental factors, thereby optimizing the harvest.

	\textbf{Car Park Management System} \\
	I created a Java-based parking management system, following the fundamentals of OOP and making use of common libraries. The console application for this project was created with scalability in mind to enable future updates and upgrades.

	\textbf{Temperature Controlled Cooling Fan} \\
	I lead a group of three in designing and implementing a prototype of temperature controlled cooling fan using PIC16F877A microcontroller. My primary role was to design the architecture for the hardware and to program the microcontroller.

	\textbf{Filter Design} \\
	Designed and implemented a low pass, band pass, and high pass filter of butterworth approximation using op-amps

	\textbf{Audio Amplifier} \\
	Designed and developed an audio amplifier using BJT transistors and FET transistors to obtain a given gain.

\end{rSection}

%----------------------------------------------------------------------------------------
%	RESEARCH
%----------------------------------------------------------------------------------------

%\begin{rSection}{Research}

	%Section content\ldots

%\end{rSection}

%----------------------------------------------------------------------------------------
%	PUBLICATIONS
%----------------------------------------------------------------------------------------

\begin{rSection}{Publications}

	% international conference
	\textbf{SLIT International Conference on Eng \& Technology } \hfill  \textit{Pending Publication}\\
	%\begin{quote}
	Perera, P.; Ahamed, A.; Amantha, K., ``A unified precomputation algorithm to decode and transverse an ONNX graph," \emph{SLIT International Conference on Eng \& Technology}, 2024 (pending publication)
	%\end{quote}


\end{rSection}

%----------------------------------------------------------------------------------------
%	PROFESSIONAL AFFILIATIONS
%----------------------------------------------------------------------------------------

\begin{rSection}{Professional Affiliations}

	\textbf{IEEE} \hfill \textit{2021 - 2023} \\ 
	Status : Student Member
	% Membership No.: xxxxxxxx

\end{rSection}

%----------------------------------------------------------------------------------------
%	ACHIEVEMENTS
%----------------------------------------------------------------------------------------

% \begin{rSection}{Achievement}

% 	\textbf{1\textsuperscript{st} place - Black Belt Male-Kumite event} \hfill \textit{2016} \\ 
% 	Gained 1\textsuperscript{st} place at the Black Belt Male-Kumite event organized by Kensho Karate International Sri Lanka Karate Do Federation (approved by the Ministry of Sports).

% 	\textbf{Best mini project for power electronics} \hfill \textit{2017} \\ 
% 	Best mini project for power electronics at the Department of Electrical and Electronics Engineering, South Eastern University, Sri Lanka.

% 	\textbf{Best student award - SMIDF} \hfill \textit{2009} \\ 
% 	Best student award for \emph{Computer Hardware} awarded by the SMIDF (Small \& Medium Industrial Development Foundation), Kururnegala.

% \end{rSection}

%----------------------------------------------------------------------------------------
%	LEADERSHIP AND TEAMWORK
%----------------------------------------------------------------------------------------

%\begin{rSection}{RESEARCH}

	%Section content\ldots

%\end{rSection}

%----------------------------------------------------------------------------------------
%	EXTRA CURRICULAR ACTIVITIES
%----------------------------------------------------------------------------------------

% \begin{rSection}{Extra Curricular Activities}

% 	\textbf{Member of University Badminton team} \hfill \textit{2014 - 2018}

% 	\textbf{Member of University Taekwondo team} \hfill \textit{2014 - 2018}

% 	\textbf{Member of University Media club} \hfill \textit{2014 - 2018}

% \end{rSection}








%----------------------------------------------------------------------------------------
%	TECHNICAL STRENGTHS SECTION
%----------------------------------------------------------------------------------------

% \begin{rSection}{Technical Strengths}

% 	\begin{tabular}{@{} >{\bfseries}l @{\hspace{6ex}} l @{}}
% 		Computer Languages & Prolog, Haskell, AWK, Erlang, Scheme, ML \\
% 		Protocols \& APIs & XML, JSON, SOAP, REST \\
% 		Databases & MySQL, PostgreSQL, Microsoft SQL \\
% 		Tools & SVN, Vim, Emacs
% 	\end{tabular}

% \end{rSection}

%----------------------------------------------------------------------------------------
%	EXAMPLE SECTION
%----------------------------------------------------------------------------------------

%\begin{rSection}{Section Name}

	%Section content\ldots

%\end{rSection}

%----------------------------------------------------------------------------------------

\end{document}
