%%%%%%%%%%%%%%%%%%%%%%%%%%%%%%%%%%%%%%%%%
% Medium Length Professional CV
% LaTeX Template
% Version 3.0 (December 17, 2022)
%
% This template originates from:
% https://www.LaTeXTemplates.com
%
% Author:
% Vel (vel@latextemplates.com)
%
% Original author:
% Trey Hunner (http://www.treyhunner.com/)
%
% License:
% CC BY-NC-SA 4.0 (https://creativecommons.org/licenses/by-nc-sa/4.0/)
%
%%%%%%%%%%%%%%%%%%%%%%%%%%%%%%%%%%%%%%%%%

%----------------------------------------------------------------------------------------
%	PACKAGES AND OTHER DOCUMENT CONFIGURATIONS
%----------------------------------------------------------------------------------------

\documentclass[
	%a4paper, % Uncomment for A4 paper size (default is US letter)
	11pt, % Default font size, can use 10pt, 11pt or 12pt
]{./assets/resume} % Use the resume class

% \usepackage{ebgaramond} % Use the EB Garamond font
\usepackage{helvet}


%------------------------------------------------

\name{Kavinda Ravishan} % Your name to appear at the top

\phone{+94773166850}
\linkedin{https://www.linkedin.com/in/kavinda-ravishan}
\email{sales@accelr.net}


% You can use the \address command up to 3 times for 3 different addresses or pieces of contact information
% Any new lines (\\) you use in the \address commands will be converted to symbols, so each address will appear as a single line.

\address{Email \\ sales@accelr.net} % Email

\address{WhatsApp \\ +94 (0)773166850} % WhatsApp Number

\address{Linkedin \\ https://www.linkedin.com/in/kavinda-ravishan} % LinkedIn Profile

%----------------------------------------------------------------------------------------

\begin{document}


% \begin{tabularx}{\textwidth}{
% 	| >{\raggedright\arraybackslash}X 
% 	| >{\raggedleft\arraybackslash}X | }
% 	\hline
% 	{\huge\bf Kasun Buddhi} \\
% 	WhatsApp : Linkedin : Email
	
% 	& \raisebox{-\totalheight}{\includegraphics[width=0.3\textwidth]{logo.png}} \\
% 	\hline
% \end{tabularx} 

% \begin{tabularx}{\textwidth}{ |X|X| } 
% 	\hline
% 	cell3 & \multirow{3}{5cm}{Multiple row} \\ 
% 	cell6 &  \\ 
% 	cell9 &  \\ 
% 	\hline
% \end{tabularx}

% \begin{tabularx}{\textwidth}{
% 	 	 >{\raggedright\arraybackslash}X 
% 	 	 >{\raggedleft\arraybackslash}X  } 
% 	\smallskip
% 	{\huge\bf Kasun Buddhi} & 
% 	\multirow[c]{3}{*}{{\includegraphics[width=0.25\textwidth]{logo.png}}}\\ 
% 	WhatsApp : Linkedin : Email & \\
% \end{tabularx}

%----------------------------------------------------------------------------------------
%	TECHNICAL STRENGTHS SECTION
%----------------------------------------------------------------------------------------

\begin{rSection}{Technical Strengths}

	\def\arraystretch{1.5}

	\begin{tabular}{ l l}
		\textbf{Expertise} & \emph{Concurrent Programming, C++ performance acceleration} \\
		\textbf{Programming Languages} & \emph{C++, C, Python, CUDA} \\
		\textbf{Tools and Frameworks} & \emph{PyTorch (C++, Python), OpenCV (C++, Python),} \\ 
        & \emph{ARM performance libs, Intel OneAPI, LLVM, Pytest, Matlab} \\ 
		\textbf{Languages} & \emph{Sinhala-Native, English-Excellent} \\
	\end{tabular}

\end{rSection}

%----------------------------------------------------------------------------------------
%	EDUCATION SECTION
%----------------------------------------------------------------------------------------

\begin{rSection}{Education}

	\textbf{Sri Lanka Technological Campus} \hfill \textit{2016 - 2020} \\ 
    B.Sc (Hons) in Electronic \& Telecommunication Engineering
	% Overall GPA: 3.44/4.0 \\
	% Status: Second Upper
	
\end{rSection}

%----------------------------------------------------------------------------------------
%	WORK EXPERIENCE SECTION
%----------------------------------------------------------------------------------------

\begin{rSection}{Experience}

    \begin{rSubsectionX}{ACCELR}{www.accelr.lk}{Software Engineer}{Aug 2022 - Present}
		\item Developed a compiler front-end for a simple grammar using LLVM, in order to enhance my understanding of compiler design and LLVM infrastructure.
		\item Completed a course on \href{https://www.edx.org/learn/computer-science/stanford-university-compilers}{Compilers from Stanford Online} on edX, and gained understanding in lexical analysis, parsing, semantic analysis, type checking, and code generation.
		\item Compiled and successfully ran a predefined video pipeline in Google's MediaPipe framework,
	\end{rSubsectionX}

	\begin{rSubsectionX}{Analog Inference}{https://www.analog-inference.com/}{Engineering Consultant}{Aug 2022 - Present}
		\item member of the Analog Inference back-end software development team primarily responsible for optimizing host side data pipeline AI system software stack.
        \item Improved performance of host bound layers of ML models by using high-performance libraries, multi-threading, and custom implementations.
        \item Designed and developed a plugin system that can be used in both C++ and Python (using the Python/C API) for host applications. This would allow end users to easily plug-in model specific pre-processing and post-processing steps to the data pipeline.
        \item Developed and documented a C++ style guide to improve code safety, maintainability, readability, and consistency among team members. 
        \item Designed and developed an inter-process communication library to send and receive variable-length data between different processes in the host application.
	\end{rSubsectionX}

	\begin{rSubsectionX}{Zebra Technologies}{https://www.zebra.com/}{Software Engineer}{Dec 2021 - Aug 2022}
		\item Implemented robust test automation scripts for Zebra barcode scanners using pytest.
	\end{rSubsectionX}

	\begin{rSubsectionX}{Advanced Engineering Technologies (Pvt) Ltd}{https://adentsl.com/}{Intern - Software Engineer}{Mar 2019 - Sep 2019}
		\item Design and development of image processing algorithms (OpenCV) and training of ML models (Yolo) for industrial defect detection solutions.
	\end{rSubsectionX}
    
\end{rSection}

%----------------------------------------------------------------------------------------
%	PROJECTS
%----------------------------------------------------------------------------------------

\begin{rSection}{Projects}

	\textbf{Analysis of Polarization Mode Dispersion in Fiber Optics} \\
	As final year project of the B.Sc. degree program, designed and developed a simple GUI application to connect with multiple test equipment (laser source and polarimeter) through the GPIB bus, communicate with equipment using the National Instruments API, and connect with an implemented polarization controller through the USB interface and implemented an algorithm to find polarization mode dispersion with Jones matrix eigenanalysis and characterize the PMD profile of a given fiber cable.

	\textbf{Real-time Object Tracking System} \\
	Developed a C++ software for tracking objects (using the OpenCV library) and sending coordinate data via serial port to an Arduino device. This Arduino device was programmed to control server motors and to keep the detected object in the center of the screen.

	\textbf{Real-Time Chat App} \\
	A simple browser based chat application using HTML, CSS, JavaScript for front-end and NodeJS and Socket.IO for back-end. Docker container available. 
    \href{https://hub.docker.com/r/slspider/chatapp}{See Link}

\end{rSection}

%----------------------------------------------------------------------------------------
%	RESEARCH
%----------------------------------------------------------------------------------------

%\begin{rSection}{Research}

	%Section content\ldots

%\end{rSection}

%----------------------------------------------------------------------------------------
%	PUBLICATIONS
%----------------------------------------------------------------------------------------

%\begin{rSection}{Publications}

	%Section content\ldots

%\end{rSection}

%----------------------------------------------------------------------------------------
%	PROFESSIONAL AFFILIATIONS
%----------------------------------------------------------------------------------------

% \begin{rSection}{Professional Affiliations}

% 	\textbf{Institution of Engineers, Sri Lanka (IESL)} \hfill \textit{Since 2019} \\ 
% 	Status : Student Member \\
% 	Membership No.: S-26740

% 	\textbf{IEEE} \hfill \textit{Since 2019} \\ 
% 	Status : Student Member \\
% 	% Membership No.: xxxxxxxx

% \end{rSection}

%----------------------------------------------------------------------------------------
%	ACHIEVEMENTS
%----------------------------------------------------------------------------------------

% \begin{rSection}{Achievement}

% 	\textbf{1\textsuperscript{st} place - ASRITE 2017} \hfill \textit{2017} \\ 
% 	Final year research project demonstration of B.Sc. Special degree program. Annual Symposium on Research and Industrial Training of Department of Electronics 2017.

% 	\textbf{CHAMPIONS - SAITM Robotics Challenge} \hfill \textit{2015} \\ 
% 	organized by SAITM Malabe Campus.

% 	\textbf{CHAMPIONS - Technosoft} \hfill \textit{2015} \\ 
% 	Robotics Application Development Competition Organized by Advanced Technological Institute Kurunegala.

% \end{rSection}

%----------------------------------------------------------------------------------------
%	LEADERSHIP AND TEAMWORK
%----------------------------------------------------------------------------------------

%\begin{rSection}{RESEARCH}

	%Section content\ldots

%\end{rSection}

%----------------------------------------------------------------------------------------
%	EXTRA CURRICULAR ACTIVITIES
%----------------------------------------------------------------------------------------

% \begin{rSection}{Extra Curricular Activities}

% 	\textbf{President of the Electronics Society (E-soc), WUSL} \hfill \textit{2014 - 2015}

% 	\textbf{Vice President of the Electronics Society (E-soc), WUSL} \hfill \textit{2013 - 2014}

% 	\textbf{Member of Computer Society (ACISS), WUSL} \hfill \textit{2013 - 2017}

% \end{rSection}








%----------------------------------------------------------------------------------------
%	TECHNICAL STRENGTHS SECTION
%----------------------------------------------------------------------------------------

% \begin{rSection}{Technical Strengths}

% 	\begin{tabular}{@{} >{\bfseries}l @{\hspace{6ex}} l @{}}
% 		Computer Languages & Prolog, Haskell, AWK, Erlang, Scheme, ML \\
% 		Protocols \& APIs & XML, JSON, SOAP, REST \\
% 		Databases & MySQL, PostgreSQL, Microsoft SQL \\
% 		Tools & SVN, Vim, Emacs
% 	\end{tabular}

% \end{rSection}

%----------------------------------------------------------------------------------------
%	EXAMPLE SECTION
%----------------------------------------------------------------------------------------

%\begin{rSection}{Section Name}

	%Section content\ldots

%\end{rSection}

%----------------------------------------------------------------------------------------

\end{document}
