%%%%%%%%%%%%%%%%%%%%%%%%%%%%%%%%%%%%%%%%%
% Medium Length Professional CV
% LaTeX Template
% Version 3.0 (December 17, 2022)
%
% This template originates from:
% https://www.LaTeXTemplates.com
%
% Author:
% Vel (vel@latextemplates.com)
%
% Original author:
% Trey Hunner (http://www.treyhunner.com/)
%
% License:
% CC BY-NC-SA 4.0 (https://creativecommons.org/licenses/by-nc-sa/4.0/)
%
%%%%%%%%%%%%%%%%%%%%%%%%%%%%%%%%%%%%%%%%%

%----------------------------------------------------------------------------------------
%	PACKAGES AND OTHER DOCUMENT CONFIGURATIONS
%----------------------------------------------------------------------------------------

\documentclass[
	%a4paper, % Uncomment for A4 paper size (default is US letter)
	11pt, % Default font size, can use 10pt, 11pt or 12pt
]{./../assets/resume} % Use the resume class

% \usepackage{ebgaramond} % Use the EB Garamond font
\usepackage{helvet}


%------------------------------------------------

\name{Praveen Alahakoon} % Your name to appear at the top

\phone{+94773166850}
\linkedin{https://www.linkedin.com/in/praveen-alahakoon/}
\email{sales@accelr.net}


% You can use the \address command up to 3 times for 3 different addresses or pieces of contact information
% Any new lines (\\) you use in the \address commands will be converted to symbols, so each address will appear as a single line.

\address{Email \\ sales@acceler.net} % Email

\address{WhatsApp \\ +94 (0)71 087 3080} % WhatsApp Number

\address{Linkedin \\ https://www.linkedin.com/in/praveen-alahakoon/} % LinkedIn Profile

%----------------------------------------------------------------------------------------

\begin{document}


% \begin{tabularx}{\textwidth}{
% 	| >{\raggedright\arraybackslash}X 
% 	| >{\raggedleft\arraybackslash}X | }
% 	\hline
% 	{\huge\bf Kasun Buddhi} \\
% 	WhatsApp : Linkedin : Email
	
% 	& \raisebox{-\totalheight}{\includegraphics[width=0.3\textwidth]{logo.png}} \\
% 	\hline
% \end{tabularx} 

% \begin{tabularx}{\textwidth}{ |X|X| } 
% 	\hline
% 	cell3 & \multirow{3}{5cm}{Multiple row} \\ 
% 	cell6 &  \\ 
% 	cell9 &  \\ 
% 	\hline
% \end{tabularx}

% \begin{tabularx}{\textwidth}{
% 	 	 >{\raggedright\arraybackslash}X 
% 	 	 >{\raggedleft\arraybackslash}X  } 
% 	\smallskip
% 	{\huge\bf Kasun Buddhi} & 
% 	\multirow[c]{3}{*}{{\includegraphics[width=0.25\textwidth]{logo.png}}}\\ 
% 	WhatsApp : Linkedin : Email & \\
% \end{tabularx}

%----------------------------------------------------------------------------------------
%	TECHNICAL STRENGTHS SECTION
%----------------------------------------------------------------------------------------

\begin{rSection}{Technical Strengths}

	\def\arraystretch{1.5}

	\begin{tabular}{ l l}
		\textbf{Expertise} & \emph{RTL Design, CPU Design, Parallel Computing, ML Acceleration} \\
		\textbf{Programming Languages} & \emph{C++, Python, Verilog} \\
		\textbf{Tools and Frameworks} & \emph{Xilinx Vivado, Apache TVM} \\ 
		\textbf{Languages} & \emph{Sinhala-Native, English-Good} \\
	\end{tabular}

\end{rSection}

%----------------------------------------------------------------------------------------
%	EDUCATION SECTION
%----------------------------------------------------------------------------------------

\begin{rSection}{Education}

	\textbf{Master of Philosophy (MPhill), University of Ruhuna, Sri Lanka} \hfill \textit{2019} \\ 
	Course Content - Design and implement a novel processor architecture for accelerating CNN calculations using Field Programmable Gate Array (FPGA) \\
	Status : Publication writing

	\textbf{University of Ruhuna, Sri Lanka} \hfill \textit{2015 - 2018} \\ 
	Bachelor of Science (B.Sc.) (Hons)\\
	Overall GPA: 3.03/4.0 \\
	Status: Second Lower
	
\end{rSection}

%----------------------------------------------------------------------------------------
%	WORK EXPERIENCE SECTION
%----------------------------------------------------------------------------------------

\begin{rSection}{Experience}

	\begin{rSubsectionX}{ACCELR}{www.accelr.lk}{Electronic Engineer}{Feb 2023 - Present}
		\item In-depth study of the RISC-V ISA and memory model including memory ordering, caching and virtual memory.
	\end{rSubsectionX}

	\begin{rSubsectionX}{Quadric.io}{www.quadric.io}{Consultant - Software Engineer}{Mar 2023 - Feb 2024}
        \item Was a member of Quadric's application team and was responsible for developing high-performance and parallel computing applications, including neural networks, utilizing Quadric's parallel GPNPU architecture.
        \item Contributed to building \href{https://quadric.io/sdk/}{Software Development Kit (SDK)} for Quadric's Chimera accelerator
        \item Improved the (TVM based) Quadric graph compiler efficiency by introducing Quadric SDK functions to it.
    \end{rSubsectionX}

	\begin{rSubsectionX}{University of Ruhuna, Sri Lanka}{www.ruh.ac.lk}{Research Assistant}{Mar 2019 - Mar 2023}
		\item Designing and implementation of novel processor architecture for accelerating the CNN calculations
		\item Implementation of novel Instruction Set Architecture for a soft-core CPU
		\item Designing and implementation of special cache memory technology for CNN data access
		\item Implementation of basic compiler environment for the ISA using Python
		\item Application for the patent (Status: Pending)
	\end{rSubsectionX}

	% \pagebreak

	\begin{rSubsectionX}{University of Ruhuna, Sri Lanka}{www.ruh.ac.lk}{Temporary Demonstrator}{Feb 2018 - Feb 2019}
		\item Designing and implementation of Automatic Shading mechanism for a Camera
		\item Obtaining a patent \href{https://patentscope.wipo.int/search/en/detail.jsf?docId=WO2021069993&_cid=P20-LSKB9J-42754-1}{for the IP}
		\item Application of Patent Cooperation Treaty (PCT) for the patent 
	\end{rSubsectionX}

\end{rSection}

%----------------------------------------------------------------------------------------
%	PROJECTS
%----------------------------------------------------------------------------------------

\begin{rSection}{Projects}

	\textbf{Reconfigurable Convolutional Neural Network Processor (RCNNP)} \\
	Novel Processor architecture for CNN inference acceleration with unique memory cache technology. novel ISA, assembly language and assembler program. The design was verified using an FPGA hardware platform. \\ 
	Local patent (Pending) and PCT application submitted and ISR received 


	\textbf{Sri Lanka Meteor Network} \\
	Established the All-sky camera network in Sri Lanka to monitor meteor's entering the atmosphere.
	Developed novel communication protocol and CNN meteor detection algorithm.\\
	Received two international copyrights, a local patent and PCT application submitted and ISR received 

	\textbf{Emergency Messaging and Data Communication Network} \\
	Designed and implemented a new secure network and transport layer communication protocol to communicate using over RF transceivers during disaster situations when traditional and cellular network coverage is absent.

	\textbf{Automated Solar Tracking Telescope} \\
	Developed a solar tracking telescope that tracks to position of the sun using image processing techniques (OpenCV). System was implemented on a Raspberry pi 2 B+ SBC. \\
	Gold medal from University Politechnica of Bucharest, Romania \\
	Gold medal from “Sahasak Nimawum 2017”, Colombo, Sri Lanka \\
	Silver medal from “IPIITEX 2018”, Bangkok, Thailand

\end{rSection}

%----------------------------------------------------------------------------------------
%	RESEARCH
%----------------------------------------------------------------------------------------

% \begin{rSection}{Research}

% 	Section content
% 	\ldots

% \end{rSection}

%----------------------------------------------------------------------------------------
%	PUBLICATIONS
%----------------------------------------------------------------------------------------

\begin{rSection}{Publications}

	\textbf{Unveiling an Automated Modeling Method for S0 Galaxies} \\
	{Research Notes of the AAS}{Jun 2, 2021}
	\href{https://iopscience.iop.org/article/10.3847/2515-5172/ac0dc0}{see publication}

	\textbf{Conversion of the Convolutional Neural Network implemented on Keras API to plain Python script} \\
	{8th Ruhuna International Science and technology Conference}{Feb 17, 2021}
	\href{https://www.researchgate.net/publication/360994589_Conversion_of_the_Convolutional_Neural_Network_implemented_on_Keras_API_to_plain_Python_script}{see publication}

	\textbf{Automated Modelling Method for S0 galaxies to Unveil the Effect of Cluster Environment on their Formation} \\
	{237th meeting of the American Astronomical Society}{Jan 1, 2021}
	\href{https://www.researchgate.net/publication/348993949_Automated_Modelling_Method_for_S0_galaxies_to_Unveil_the_Effect_of_Cluster_Environment_on_their_Formation}{see publication}
	
	\textbf{Characterization of Aluminium Reflective Thin Films Deposited by aModified Thermal Evaporator Designed for Coating Telescope Optics} \\
	{Institute of Physics – Sri Lanka}{Aug 1, 2020}
	\href{https://www.researchgate.net/publication/344312143_Characterization_of_Aluminium_Reflective_Thin_Films_Deposited_by_a_Modified_Thermal_Evaporator_Designed_for_Coating_Telescope_Optics}{see publication}

	\textbf{Itraframe image processing algorithm for identifying meteors in all-sky images} \\
	{7th Ruhuna International Science and Technology Conference}{Jan 23, 2020}

	\textbf{Unveiling an optimum method for modeling cluster S0 galaxies} \\
	{7th Ruhuna International Science and Technology Conference (RISTCON 2020)}{Jan 1, 2020}

	\textbf{A Robotic Camera for Monitoring Meteors Entering the Earth’s Atmosphere near the Equator} \\
	{Advances in Astrophysics}{Aug 3, 2019}
	\href{http://www.isaacpub.org/images/PaperPDF/AdAp_100132_2019082716321960559.pdf}{see publication}

	\textbf{Development of an autonomous wide-field camera unit for monitoring optical transients that occur in the atmosphere} \\
	{Conference: Space Technology}{Dec 1, 2018}
	\href{https://www.researchgate.net/publication/332246125_Development_of_an_autonomous_wide-field_camera_unit_for_monitoring_optical_transients_that_occur_in_the_atmosphere}{see publication}

\end{rSection}

\pagebreak

%----------------------------------------------------------------------------------------
%	PROFESSIONAL AFFILIATIONS
%----------------------------------------------------------------------------------------

% \begin{rSection}{Professional Affiliations}

% 	\textbf{Institution of Engineers, Sri Lanka (IESL)} \hfill \textit{Since 2019} \\ 
% 	Status : Student Member \\
% 	Membership No.: S-26740

% 	\textbf{IEEE} \hfill \textit{Since 2019} \\ 
% 	Status : Student Member \\
% 	% Membership No.: xxxxxxxx

% \end{rSection}

%----------------------------------------------------------------------------------------
%	ACHIEVEMENTS
%----------------------------------------------------------------------------------------

\begin{rSection}{Achievement}

	\textbf{VC's Award for the Most Outstanding Student Inventor of the year} \hfill \textit{2022} \\
	Awarded by University of Ruhuna · Jun 2023

	\textbf{Patent: Reconfigurable Convolution Neural Network Processor} \hfill \textit{2022} \\ 
	Filling number: 22505

	\textbf{Patent: Automatic Shading Mechanism for a Camera} \hfill \textit{2019} \\ 
	Filling number: 20804

	\textbf{Patent: Day and nighttime imaging capable wide-field autonomous camera unit with Internet of things technology for continuous sky monitoring} \hfill \textit{2019} \\ 
	Filling number: 20432



\end{rSection}

%----------------------------------------------------------------------------------------
%	LEADERSHIP AND TEAMWORK
%----------------------------------------------------------------------------------------

%\begin{rSection}{RESEARCH}

	%Section content\ldots

%\end{rSection}

%----------------------------------------------------------------------------------------
%	EXTRA CURRICULAR ACTIVITIES
%----------------------------------------------------------------------------------------

% \begin{rSection}{Extra Curricular Activities}

% 	\textbf{Member of University Badminton team} \hfill \textit{2014 - 2018}

% 	\textbf{Member of University Taekwondo team} \hfill \textit{2014 - 2018}

% 	\textbf{Member of University Media club} \hfill \textit{2014 - 2018}

% \end{rSection}








%----------------------------------------------------------------------------------------
%	TECHNICAL STRENGTHS SECTION
%----------------------------------------------------------------------------------------

% \begin{rSection}{Technical Strengths}

% 	\begin{tabular}{@{} >{\bfseries}l @{\hspace{6ex}} l @{}}
% 		Computer Languages & Prolog, Haskell, AWK, Erlang, Scheme, ML \\
% 		Protocols \& APIs & XML, JSON, SOAP, REST \\
% 		Databases & MySQL, PostgreSQL, Microsoft SQL \\
% 		Tools & SVN, Vim, Emacs
% 	\end{tabular}

% \end{rSection}

%----------------------------------------------------------------------------------------
%	EXAMPLE SECTION
%----------------------------------------------------------------------------------------

%\begin{rSection}{Section Name}

	%Section content\ldots

%\end{rSection}

%----------------------------------------------------------------------------------------

\end{document}
